\documentclass[../notes.tex]{subfiles}

\pagestyle{main}
\renewcommand{\chaptermark}[1]{\markboth{\chaptername\ \thechapter\ (#1)}{}}
\setcounter{chapter}{3}

\begin{document}




\chapter{Carboxylic Acids and Derivatives}
\setcounter{section}{22}
\section{Carboxylic Acids Intro}
\begin{itemize}
    \item \marginnote{10/30:}Lecture 22 recap.
    \begin{enumerate}[label={\Alph*.}]
        \item Amine synthesis by direct S\textsubscript{N}2 (of, for example, \ce{NH3}) leads to mixtures unless you use a very large excess of ammonia (Figure \ref{fig:amineAlkylationRX}).
        \begin{itemize}
            \item Alternative: Gabriel synthesis (Figure \ref{fig:gabrielSynthesis}).
            \item Alternative: Conversion of a primary or secondary alkyl halide to an azide and subsequent reduction (Figure \ref{fig:azideReduce}).
        \end{itemize}
        \item Reductive amination is an incredibly powerful technique (Figures \ref{fig:redAmin23}, \ref{fig:redAmin12}, \& \ref{fig:redAmin01}).
        \begin{itemize}
            \item It can build primary, secondary, and tertiary amines.
            \item Be intimately familiar with this process for Exam 3!!
        \end{itemize}
        \item Acylation/reduction is also a great method (Figure \ref{fig:acylReduce}).
        \begin{itemize}
            \item Acylate the amine to give an amide intermediate, reduce with LAH, and quench with water.
        \end{itemize}
        \item Primary and secondary alkyl bromides, iodides, and tosylates can be substituted to the nitrile and reduced to an amine (Figure \ref{fig:nitrileReducX}).
        \begin{itemize}
            \item This is a 1-carbon homologation.
        \end{itemize}
        \item HONO (generated from \ce{NaNO2 + HCl}) converts aniline to an aryl diazonium salt (Figure \ref{fig:diazoniumForm}).
    \end{enumerate}
    \item Announcement: The notes taken by the TFs are posted on Canvas (that's these!).
    \begin{itemize}
        \item Consider referring to these even over the ones that Prof. Buchwald provides.
    \end{itemize}
    \item Lecture 22 continued.
    \item Using the sequence of reaction in Figure \ref{fig:diazoniumBenz}, you can form an aryl diazonium salt.
    \begin{itemize}
        \item Treating it with \ce{KI} yields an aryl iodide.
        \item Treating it with \ce{H2O} yields a phenol.
        \item Treating it with hypophosphorus acid (\ce{H3PO2}) yields benzene again.
        \begin{itemize}
            \item Once again, you are not responsible for the name "hypophosphorus acid."
        \end{itemize}
        \item Treating it with \ce{CuX} (where $\ce{X}=\ce{Cl},\ce{Br},\ce{CN}$) yields \ce{PhX}.
    \end{itemize}
    \item This is a great example of what we do with synthesis!
    \begin{itemize}
        % \item Example: \emph{para}-bromoanisole maybe came from HONO followed by the aryl diazonium salt??
        \item Synthesis is all about connecting compounds with transformations.
        \item Breaking down the example in such a way is called \textbf{retrosynthetic analysis}.
    \end{itemize}
    \item Recall from last time that azides are reduced to amines by \ce{LiAlH4} and a subsequent water workup (Figure \ref{fig:azideReduce}). Here's a further note on this.
    \begin{figure}[H]
        \centering
        \footnotesize
        \schemestart
            \chemfig{*6(--(*3(<O>))----)}
            \arrow(c1--c23){0}[,2]
            \subscheme{
                \chemfig{*6(--(<OH)-(<:N_3)---)}
                \arrow(c2--c3){->[{[H]}]}[,1.4]
                \chemfig{*6(--(<OH)-(<:NH_2)---)}
                \arrow(@c2--c4){0}[-90,0.5]
                \chemfig{*6(--(<OH)-(<:CN)---)}
                \arrow(--c5){->[1. \ce{LiAlH4}][2. \ce{H2O}\hspace{3.6mm}\ ]}[,1.4]
                \chemfig{*6(--(<OH)-(<:-[:-30]NH_2)---)}
            }
        \schemestop
        \chemmove{
            \draw ([xshift=9pt]c1.east) -- ($(c1)!0.3!(c23)$) coordinate (c25) -- (c25 |- c2) -- node[above]{1. \ce{NaN3}} node[below]{2. \ce{H+}\hspace{3mm}\ } ([xshift=-9pt]c2.west);
            \draw (c25) -- (c25 |- c4) -- node[above]{1. \ce{NaCN}} node[below]{2. \ce{H+}\hspace{4mm}\ } ([xshift=-9pt]c4.west);
        }
        \caption{Aminoalcohol synthesis from epoxides.}
        \label{fig:aminoalcoholEpox}
    \end{figure}
    \begin{itemize}
        \item Recall from 5.12 that \textbf{epoxides} are essentialy just reactive ethers, due to their ring strain.
        \item Therefore, if we treat an epoxide with \ce{NaN3}, we'll get a backside attack that yields a certain intermediate.
        \item Then upon reduction, we get a \emph{trans}-1,2-aminoalcohol.
        \begin{itemize}
            \item This is an important functional group for $\beta$-blockers in biology!
        \end{itemize}
        \item Alternatively, we can treat epoxides with \ce{CN-}, yielding the cyanoalcohol.
        \begin{itemize}
            \item We can then reduce this to the 1,3-aminoalcohol.
        \end{itemize}
    \end{itemize}
    \item This concludes our discussion of amines.
    \item Today: Introduction to carboxylic acids and their derivatives.
    \begin{itemize}
        \item Reading: Chapter 10 of \textcite{bib:Clayden}.
    \end{itemize}
    \item Lecture outline.
    \begin{enumerate}
        \item Introduction.
        \item Synthesis of carboxylic acids.
        \begin{enumerate}[label={\alph*.}]
            \item Oxidation of alcohols and aldehydes.
            \item Carboxylation of Grignard reagents.
            \item Hydrolysis of nitriles.
            \item Types of carboxylic acid derivatives.
        \end{enumerate}
        \item Acyl transfer reactions.
        \begin{enumerate}[label={\alph*.}]
            \item Background.
        \end{enumerate}
    \end{enumerate}
    \item We'll begin with Topic 1: Introduction.
    \item \textbf{Carboxylic acid derivative}: A compound of the following form, where $\ce{X}\neq\ce{H},\ce{R}$. \emph{Structure}
    \begin{figure}[h!]
        \centering
        \footnotesize
        \chemfig{R-[:30](=[2]O)-[:-30]X}
        \caption{Carboxylic acid derivative.}
        \label{fig:carbDeriv}
    \end{figure}
    \begin{itemize}
        \item Since \ce{X} is \emph{not} equal to \ce{H} or \ce{R}, we're not considering aldehydes or ketones.
    \end{itemize}
    \pagebreak
    \item \textbf{Carboxylic acid}: A carboxylic acid derivative for which $\ce{X}=\ce{OH}$. \emph{Structure}
    \begin{figure}[h!]
        \centering
        \footnotesize
        \schemestart
            \chemfig{R-[:30](=[2]O)-[:-30]OH}
            \arrow{<=>[][\ce{H+}]}
            \chemleft{[}\subscheme{
                \chemfig{R-[:30](=[2]O)-[:-30]\charge{45=$\ominus$}{O}}
                \arrow{<->}
                \chemfig{R-[:30](=[:-30]O)-[2]\charge{45=$\ominus$}{O}}
            }\chemright{]}
            \arrow{0[\small$\equiv$][][-2mm]}
            \chemfig{R-[:30]\charge{[extra sep=8pt]30=$\ominus$}{}(-[2]@{Ot}O)-[:-30]@{Ob}O}
        \schemestop
        \chemmove{
            \draw [densely dashed,-,shorten <=7pt,shorten >=7pt] ($(Ot)+(30:0.05)$) to[bend right=30,looseness=1.2] ($(Ob)+(30:0.05)$);
        }
        \caption{Carboxylic acid.}
        \label{fig:carbAcid}
    \end{figure}
    \begin{itemize}
        \item $\pKa\approx 5$.
        \begin{itemize}
            \item By comparison, $\pKa\approx 16$ for an alcohol.
            \item Therefore, carboylic acids are \emph{eleven orders of magnitude} more acidic than alcohols.
        \end{itemize}
        \item Deprotonation gives us a resonance-stabilized \textbf{carboxylate}, which can be drawn either as resonance forms or as a delocalized anion.
    \end{itemize}
    \item One of the simplest carboxylic acids is \textbf{acetic acid}.
    \item \textbf{Acetic acid}: The carboxylic acid for which $\ce{R}=\ce{Me}$. \emph{Structure}
    \begin{figure}[h!]
        \centering
        \footnotesize
        \chemfig{H_3C-[:30](=[2]O)-[:-30]OH}
        \caption{Acetic acid.}
        \label{fig:aceticAcid}
    \end{figure}
    \begin{itemize}
        \item Acetic acid is in vinegar! In fact, vinegar is about 4-5\% acetic acid in water.
        \item Acetic acid is also used as an industrial solvent (in the 100\% pure form, which is quite caustic).
        \item How is acetic acid made?
        \begin{equation*}
            \ce{MeOH ->[CO][cat] CH3COOH}
        \end{equation*}
        \begin{itemize}
            \item Acetic acid is produced industrially via the Monsanto acetic acid process, which carries out the carbonylation of methanol using a rhodium catalyst.
        \end{itemize}
    \end{itemize}
    \item The first several biscarboxylic acids.
    \begin{figure}[h!]
        \centering
        \footnotesize
        \begin{subfigure}[b]{0.25\linewidth}
            \centering
            \chemfig{HO_2C-CO_2H}
            \caption{Oxalic acid.}
            \label{fig:biscarboxylica}
        \end{subfigure}
        \begin{subfigure}[b]{0.25\linewidth}
            \centering
            \chemfig{HO_2C-[:30]-[:-30]CO_2H}
            \caption{Malonic acid.}
            \label{fig:biscarboxylicb}
        \end{subfigure}
        \begin{subfigure}[b]{0.25\linewidth}
            \centering
            \chemfig{HO_2C-[:60]--[:-60]CO_2H}
            \caption{Succinic acid.}
            \label{fig:biscarboxylicc}
        \end{subfigure}\\[2em]
        \begin{subfigure}[b]{0.25\linewidth}
            \centering
            \chemfig{HO_2C-[2,,3]-[:30]-[:-30]-[6]CO_2H}
            \caption{Glutaric acid.}
            \label{fig:biscarboxylicd}
        \end{subfigure}
        \begin{subfigure}[b]{0.25\linewidth}
            \centering
            \chemfig{HO_2C-[:120,,3]-[::-60]-[::-60]-[::-60]-[::-60,,,1]CO_2H}
            \caption{Adipic acid.}
            \label{fig:biscarboxylice}
        \end{subfigure}
        \begin{subfigure}[b]{0.25\linewidth}
            \centering
            \chemfig{HO_2C-[3,,3]-[2,1.2]-[::-60,1.4]-[::-60,1.4]-[::-60,1.2]-[5,,,1]CO_2H}
            \caption{Pimelic acid.}
            \label{fig:biscarboxylicf}
        \end{subfigure}
        \caption{Biscarboxylic acids.}
        \label{fig:biscarboxylic}
    \end{figure}
    \begin{itemize}
        \item \textbf{Oxalic}, \textbf{malonic}, \textbf{succinic}, \textbf{glutaric}, \textbf{adipic}, and \textbf{pimelic} acids.
        \item Aside: Adipic acid is really important because it's involved in the manufacture of nylon.
        \item How do you remember all these names? There's a neumonic: OMSGAP or "Oh My, Such Good Apple Pie."
    \end{itemize}
    \item We now move onto Topic 2: Synthesis of carboxylic acids.
    \item Aside: A new definition of \textbf{oxidation} and \textbf{reduction}.
    \begin{itemize}
        \item Notice that in a carboxylic acid (e.g., see Figure \ref{fig:aceticAcid}), the central carbon has 3 bonds to oxygen.
        \item In contrast, a primary alcohol's central carbon has 1 bond to oxygen.
        \begin{itemize}
            \item Thus, we need to do a 4-electron oxidation to turn an alcohol into a carboxylic acid.
        \end{itemize}
        \item An aldehyde's central carbon has 2 bonds to oxygen.
        \begin{itemize}
            \item Thus, we need to do a 2-electron oxidation to turn an aldehyde into a carboxylic acid.
        \end{itemize}
        \item \ce{CO2}'s central carbon has 4 bonds to oxygen.
        \begin{itemize}
            \item Thus, we need to do a 2-electron reduction to turn \ce{CO2} into a carboxylic acid.
        \end{itemize}
        \item This array of related compounds motivates the following two definitions.
    \end{itemize}
    \item \textbf{Oxidation}: A chemical reaction that increases the number of carbon-oxygen bonds.
    \item \textbf{Reduction}: A chemical reaction that decreases the number of carbon-oxygen bonds.
    \item We now discuss Subtopic 2.a{}: Oxidation of alcohols and aldehydes.
    \begin{figure}[h!]
        \centering
        \footnotesize
        \begin{subfigure}[b]{0.57\linewidth}
            \centering
            \schemestart
                \chemfig{R-[:30]-[:-30]OH}
                \arrow{->[\ce{H2SO4}][\ce{CrO3}]}[,1.2]
                \chemleft{[}
                    \chemfig{R-[:30](=[2]O)-[:-30]H}
                \chemright{]}
                \arrow[,1.2]
                \chemfig{R-[:30](=[2]O)-[:-30]OH}
            \schemestop
            \caption{Primary alcohol or aldehyde to carboxylic acid.}
            \label{fig:OxAlcAlda}
        \end{subfigure}
        \begin{subfigure}[b]{0.42\linewidth}
            \centering
            \schemestart
                \chemfig{R-[:30]-[:-30]OH}
                \arrow{->[PCC]}[,1.2]
                \chemfig{R-[:30](=[2]O)-[:-30]H}
            \schemestop
            \caption{Primary alcohol to aldehyde.}
            \label{fig:OxAlcAldb}
        \end{subfigure}
        \caption{Oxidation of alcohols and aldehydes.}
        \label{fig:OxAlcAld}
    \end{figure}
    \begin{itemize}
        \item Suppose you have a primary alcohol.
        \begin{itemize}
            \item To convert it into a carboxylic acid, treat it with \textbf{Jones reagent}.
            \begin{itemize}
                \item The mechanism proceeds through the aldehyde.
                \item However, it can't stop, so it goes all the way to carboylic acid.
            \end{itemize}
            \item To stop the oxidation at the aldehyde, use PCC!
        \end{itemize}
        \item Now suppose you're starting at the aldehyde.
        \begin{itemize}
            \item To convert it to the carboxylic acid, just subject it to Jones reagent conditions! This is like picking up in the middle of the Figure \ref{fig:OxAlcAlda} mechanism.
        \end{itemize}
        \item Relevant reading: \textcite[194-196]{bib:Clayden}.
    \end{itemize}
    \item \textbf{Jones reagent}: The combination of excess \ce{H2SO4} and \ce{CrO3}.
    \item We now discuss Subtopic 2.b{}: Carboxylation of Grignard\footnote{"GRIN-yurd"} reagents.
    \begin{figure}[h!]
        \centering
        \footnotesize
        \schemestart
            \chemfig{R-Br}
            \arrow{->[\ce{Mg}]}
            \chemfig{R-MgBr}
            \arrow{->[\ce{CO2}]}
            \chemfig{R-[:30](=[2]O)-[:-30]\charge{45=$\ominus$}{O}}
            \arrow{->[\ce{H+}]}
            \chemfig{R-[:30](=[2]O)-[:-30]OH}
        \schemestop
        \caption{Carboxylation of Grignard reagents.}
        \label{fig:GrignardCO2}
    \end{figure}
    \pagebreak
    \begin{itemize}
        \item To make a Grignard reagent, react an alkyl bromide with magnesium.
        \begin{itemize}
            \item Aside (chemis-tea): Victor Grignard won the Nobel Prize for Grignard reagents, even though his mentor invented them!
            \item Note that Grignard reagents are very reactive! They are strong bases and strong nucleophiles, so if there's an acidic hydrogen in solution, it will get deprotonated.
            \begin{itemize}
                \item Essentially, we have to consider the functional group tolerance of a method.
            \end{itemize}
            \item These reactions are fun to do in the lab!
        \end{itemize}
        \item Once you make the Grignard reagent, just throw dry ice (a source of \ce{CO2}) into the flask. There will be a bunch of bubbling, and we'll get our carboxylic acid.
    \end{itemize}
    \item We now discuss Subtopic 2.c{}: Hydrolysis of nitriles.
    \begin{figure}[h!]
        \centering
        \footnotesize
        \schemestart
            \chemfig{R-CN}
            \arrow{->[{[O]}]}
            \chemfig{R-[:30](=[2]O)-[:-30]OH}
        \schemestop
        \caption{Nitrile hydrolysis.}
        \label{fig:nitrileHydro}
    \end{figure}
    \begin{itemize}
        \item Two ways to do this.
        \begin{itemize}
            \item Acid (\ce{H3O+}) and heat ($\Delta$).
            \item Base (\ce{HO-}), water (\ce{H2O}), and heat ($\Delta$) followed by subsequent quenching with acid and heat.
        \end{itemize}
        \item Nitriles are \emph{really, really, really} good intermediates (hint for Exam 3!!).
    \end{itemize}
    \item We'll now look at how nitriles may come up in a typical test question.
    \item Typical test question (TTQ): Provide two ways to convert benzyl bromide into phenylacetic acid.
    \begin{figure}[h!]
        \centering
        \footnotesize
        \schemestart
            \chemfig{Ph-[:30]-[:-30]Br}
            \arrow(c1--){->[*{0}\ce{Mg}]}[-45]
            \chemfig{Ph-[:30]-[:-30]MgBr}
            \arrow{->[1. \ce{CO2}][2. \ce{H+}\hspace{1.5mm}\ ]}[,1.4]
            \chemfig{Ph-[:30]-[:-30]CO_2H}
            \arrow(@c1--){->[*{0.0}\ce{NC-}]}[-135]
            \chemfig{Ph-[:30]-[:-30]CN}
            \arrow{->[Hydrolyse]}[180,1.4]
            \chemfig{Ph-[:30]-[:-30]CO_2H}
        \schemestop
        \caption{Typical test question: Multiple synthetic paths.}
        \label{fig:TTQmultSynth}
    \end{figure}
    \begin{itemize}
        \item First way: Make the Grignard and add \ce{CO2}.
        \item Second way: Do an S\textsubscript{N}2 with \ce{CN-}, and then hydrolyze the nitrile.
        \item Note that Prof. Buchwald uses checkmarks to denote the product on the board.
    \end{itemize}
    \item If we're answering a test question like this, will you want two separate arrows, or is one arrow with "1. \emph{reagent}" above and "2. \emph{reagent}" below?
    \begin{itemize}
        \item Either is good.
    \end{itemize}
    \item We now discuss Subtopic 2.d{}: Types of carboxylic acid derivatives.
    \item \textbf{Acid chloride}: A carboxylic acid derivative for which $\ce{X}=\ce{Cl}$. \emph{Structure}
    \begin{figure}[H]
        \centering
        \footnotesize
        \chemfig{R-[:30](=[2]O)-[:-30]Cl}
        \caption{Acid chloride.}
        \label{fig:carbAcidCl}
    \end{figure}
    \begin{itemize}
        \item These are far more common than acid bromides or acid iodides.\footnote{Coincidentally, acid iodides are used in the Monsanto acetic acid process!}
        \item To convert a carboxylic acid into an acid chloride, use \ce{SOCl2} and pyridine.\footnote{See the 5.12 equation review sheet!!}
        \item Mechanism: \textcite[214-215]{bib:Clayden}.
    \end{itemize}
    \item \textbf{Acid anhydride}: A carboxylic acid derivative for which $\ce{X}=\ce{RCO2}$. \emph{Structure}
    \begin{figure}[h!]
        \centering
        \footnotesize
        \chemfig{R-[:30](=[2]O)-[:-30]O-[:30](=[2]O)-[:-30]R}
        \caption{Acid anhydride.}
        \label{fig:carbAcidAnh}
    \end{figure}
    \begin{itemize}
        \item Synthesize these from two carboylic acids that combine and release water.
    \end{itemize}
    \item Example of an acid anhydride: Phthalic anhydride.
    \begin{figure}[h!]
        \centering
        \footnotesize
        \chemfig{*6(-=(*5(-(=O)-O-(=O)-))-=-=)}
        \caption{Phthalic anhydride.}
        \label{fig:phthalicAnhydride}
    \end{figure}
    \item \textbf{Ester}: A carboxylic acid derivative for which $\ce{X}=\ce{OR$'$}$. \emph{Structure}
    \begin{figure}[h!]
        \centering
        \footnotesize
        \chemfig{R-[:30](=[2]O)-[:-30]OR'}
        \caption{Ester.}
        \label{fig:carbEster}
    \end{figure}
    \begin{itemize}
        \item Esters are common in scents and smells.
    \end{itemize}
    \item Example of an ester: Isoamyl acetate.
    \begin{figure}[h!]
        \centering
        \footnotesize
        \chemfig{-[:30](-[2])-[:-30]-[:30]-[:-30]O-[:30](=[2]O)-[:-30]}
        \caption{Isoamyl acetate.}
        \label{fig:isoamylAcetate}
    \end{figure}
    \begin{itemize}
        \item This is the odor of banana oil! The infinite corridor smells like this because of the Banana Lounge.
        \item There are easy ways to make this chemical that can legally be described as natural, even if it did not come from a banana.
    \end{itemize}
    \pagebreak
    \item \textbf{Lactone}: A cyclic ester. \emph{Example}
    \begin{figure}[h!]
        \centering
        \footnotesize
        \chemfig{*5([:18]---(=O)-O-)}
        \caption{$\gamma$-butyrolactone.}
        \label{fig:carbLactone}
    \end{figure}
    \item \textbf{Amide}: A carboxylic acid derivative for which $\ce{X}=\ce{NR$'$R$''$}$. \emph{Structure}
    \begin{figure}[h!]
        \centering
        \footnotesize
        \chemfig{R-[:30](=[2]O)-[:-30]NR'R''}
        \caption{Amide.}
        \label{fig:carbAmide}
    \end{figure}
    \item Example of a (poly)amide: Nylon.
    \begin{figure}[h!]
        \centering
        \footnotesize
        \chemfig{-[@{1,0.25}:30]\chemabove{N}{H}-[:-30]-[:30]-[:-30]-[:30]-[:-30]-[:30]-[:-30]\chembelow{N}{H}-[:30](=[2]O)-[:-30]-[:30]-[:-30]-[:30]-[:-30](=[6]O)-[@{2,0.75}:30]}
        \polymerdelim[delimiters={[]},height=9mm,depth=9mm]{1}{2}
        \caption{Nylon.}
        \label{fig:nylon}
    \end{figure}
    \item \textbf{Lactam}: A cyclic amide. \emph{Example}
    \begin{figure}[h!]
        \centering
        \footnotesize
        \chemfig{*5([:18]---(=O)-HN-)}
        \caption{2-Pyrrolidone.}
        \label{fig:carbLactam}
    \end{figure}
    \begin{itemize}
        \item Lactams are incredibly imporant; many of us are only alive because of lactams.
    \end{itemize}
    \item Examples of lactams: The penicillins, a class of molecules that changed the world.
    \begin{figure}[h!]
        \centering
        \footnotesize
        \chemfig{*4((=O)-N(*5(-(<:CO_2H)-(-[:20]Me)(-[:-20]Me)-S-))-(<:[2]H)-(<\chemabove{N}{H}-[::60](=[::60]O)-[::-60]R)-)}
        \caption{Penicillin core structure.}
        \label{fig:penicillin}
    \end{figure}
    \begin{itemize}
        \item Varying \ce{R} yields different penicillins; all penicillins share the core motif above, though.
        \item Penicillins were discovered by Alexander Flemming and changed the course of the world wars.
        \item Penicillin and amoxycillin are both $\beta$-lactam antibiotics.
    \end{itemize}
    \item We now move onto Topic 3: Acyl transfer reactions.
    \item Subtopic 3.a{}: Background.
    \item For each \ce{X} group in a carboxylic acid derivatives, let's see how good of a leaving group it is.
    \begin{table}[h!]
        \centering
        \small
        \renewcommand{\arraystretch}{1.2}
        \begin{tabular}{r|ccccc}
            \textbf{\ce{X}} & \ce{Cl} & \ce{RCO2} & \ce{OR} & \ce{NR2} & \ce{O-}\\
            \textbf{$\bm{\textbf{p}K_\textbf{a}}$ (\ce{HX})} & $-7$ & $5$ & $16$ & $\approx 35$ & VERY HIGH\\
        \end{tabular}
        \caption{Leaving groups in carboxylic acid derivatives.}
        \label{tab:carbLG}
    \end{table}
    \begin{itemize}
        \item To be clear, we're measuring the $\pKa$'s of the following reactions.
        \begin{equation*}
            \ce{HX + H2O <=> X- + H3O+}\tag*{$\Ka=?$}
        \end{equation*}
        \begin{itemize}
            \item Example: \ce{HCl + H2O <=> Cl- + H3O+}.
            \item Example: \ce{HO- + H2O <=> O^2- + H3O+}.
        \end{itemize}
        \item $\pKa$ --- a theromodynamic parameter --- is a good measure of how good of a leaving group something is.
        \begin{itemize}
            \item Important because acyl transfer reactions involve an \ce{X} group from Table \ref{tab:carbLG} departing.
            \item Thus, knowing how stable the \ce{X} group is after leaving as a conjugate base in an acid reaction can help us predict how stable it will be as a departed nucleophile in an acyl transfer reaction, and hence how likely a proposed acyl transfer reaction is to proceed.
        \end{itemize}
    \end{itemize}
    \item Let's now investigate the resonance stabilization of each of our carboxylic acid derivatives.
    \begin{figure}[h!]
        \centering
        \begin{tikzpicture}
            \footnotesize
            \node (a) {
                \schemestart
                    \chemfig{R-[:30](=[2]O)-[:-30]Cl}
                    \arrow{<->}[-90]
                    \chemfig{R-[:30](-[2]\charge{45=$\ominus$}{O})=[:-30]\charge{45=$\oplus$}{Cl}}
                \schemestop
            };
            \node (b) [right=5mm of a] {
                \schemestart
                    \chemfig{R-[:30](=[2]O)-[:-30]O-[:30](=[2]O)-[:-30]R}
                    \arrow{<->}[-90]
                    \chemfig{R-[:30](-[2]\charge{45=$\ominus$}{O})=[:-30]\charge{[extra sep=5pt]90=$\oplus$}{O}-[:30](=[2]O)-[:-30]R}
                \schemestop
            };
            \node (c) [right=5mm of b] {
                \schemestart
                    \chemfig{R-[:30](=[2]O)-[:-30]OR}
                    \arrow{<->}[-90]
                    \chemfig{R-[:30](-[2]\charge{45=$\ominus$}{O})=[:-30]\charge{[extra sep=5pt]90=$\oplus$}{O}R}
                \schemestop
            };
            \node (d) [right=5mm of c] {
                \schemestart
                    \chemfig{R-[:30](=[2]O)-[:-30]NR_2}
                    \arrow{<->}[-90]
                    \chemfig{R-[:30](-[2]\charge{45=$\ominus$}{O})=[:-30]\charge{[extra sep=5pt]90=$\oplus$}{N}R_2}
                \schemestop
            };
            \node (e) [right=5mm of d] {
                \schemestart
                    \chemfig{R-[:30](=[2]O)-[:-30]\charge{45=$\ominus$}{O}}
                    \arrow{<->}[-90]
                    \chemfig{R-[:30](-[2]\charge{45=$\ominus$}{O})=[:-30]O}
                \schemestop
            };
    
            \node (res) [below left=4.2426mm of a] {Resonance:};
            \node [below=3mm of a] {very bad};
            \node [below=3mm of b] {bad};
            \node [below=3mm of c] {ok};
            \node [below=3mm of d] {very good};
            \node [below=3mm of e] {awesome};
    
            \node (stab) [below=6mm of res.east,anchor=east] {Stabilization:};
            \node (l1)   [below=9mm of a] {low};
            \node (h1)   [below=9mm of e] {high}
                edge [<->] (l1)
            ;
    
            \node (react) [below=6mm of stab.east,anchor=east] {Reactivity:};
            \node (h2)   [below=15mm of a] {high};
            \node (l2)   [below=15mm of e] {low}
                edge [<->] (h2)
            ;
        \end{tikzpicture}
        \caption{Resonance stabilization of carboxylic acid derivatives.}
        \label{fig:carbResonance}
    \end{figure}
    \begin{itemize}
        \item The lone pairs on chlorine are high energy, so we can get some degree of resonance, but the resonance structure is very bad.\footnote{Think about MOs! Big energy difference means bad mixing and hence poor conjugation}
        \item Keep in mind that we have "awesome" resonance \emph{only} for the deprotonated, carboxylate form of a carboxylic acid; carboxylic acids, themselves, aren't nearly as stabilized.
        \item Stability and reactivity are clearly inversely related; it should make sense that the less stable something is, the more reactive it is!
    \end{itemize}
    \item From Table \ref{tab:carbLG} and Figure \ref{fig:carbResonance}, we can see that the better leaving groups form more reactive carboxylic acid derivatives, and vice versa!
\end{itemize}




\end{document}