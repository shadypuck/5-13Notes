\documentclass[../notes.tex]{subfiles}

\pagestyle{main}
\renewcommand{\chaptermark}[1]{\markboth{\chaptername\ \thechapter\ (#1)}{}}
\setcounter{chapter}{2}

\begin{document}




\chapter{Amines}
\setcounter{section}{19}
\section{Special Topics}
\begin{itemize}
    \item \marginnote{10/23:}Grade cutoffs on Exam 2.
    \begin{itemize}
        \item A-B cutoff: 80.
        \item B-C cutoff: 60.
        \item C-D cutoff: 45
        \item Only F's were people who did not take the exam.
        \item This was a significantly harder exam; y'all have been crushing it so far.
        \item Remember that these grades are meant to give you a perspective for what you're on track for; they are \emph{not} binding!
    \end{itemize}
    \item Notes on Steve Buchwald.
    \begin{itemize}
        \item He's a real big-name chemist: Has his name on a ton of reactions, can make a ton of pharmaceutical drugs, does a lot of consulting for chemical companies, etc.
        \item But also super kind, humble, and nice.
        \item Knows a ton, but is very down-to-earth and approachable.
    \end{itemize}
    \item The rest of this course will be much more synthesis-heavy.
    \begin{itemize}
        \item Feel free to continue to reach out to Prof. Elkin even though she's no longer at the blackboards!
    \end{itemize}
    \item Today: We'll have fun and talk about machine learning.
    \begin{itemize}
        \item Prof. Elkin will go through \textcite{bib:ML}, a paper about using machine learning to predict the outcome of Diels-Alder reactions.
    \end{itemize}
    \item The basic idea of what the authors are saying is that if you encode the substituents, you get good prediction of the outputs!
    \begin{itemize}
        \item Your computer doesn't know what a molecule is, so you have to encode your molecule in a way that is meaningful to a computer.
        \item For example: You should not encode benzene with alternating single- and double bonds; benzene has six equivalent bonds due to resonance!
    \end{itemize}
    \item Nowadays, computers can predict biological activities (doesn't work perfectly yet, though great progress), solubility and crystal structures (works fine), NMR spectra (works awesome), etc.
    \begin{itemize}
        \item Predicting optimal reaction conditions works awesome.
    \end{itemize}
    \item Predicting reaction outcomes or yields can be hit or miss.
    \item There have been maybe \num{1000000} chemical reactions ever catalogued, but most of them are not that useful.
    \item The low-data regime of predictive modeling is the final frontier, and the especially important one for chemistry.
    \item Taking high-level expertise and making it algorithmically applicable can be really difficult.
    \item "High accuracies are achieved only if the machine is provided some chemical `insight' about the reaction (in particular, information about the reaction's core and key substituents)."
    \item While ML models cannot provide the generality of quantum mechanics, they work much faster.
    \item They trained the model with inverse electron-demand Diels-Alder reactions, Diels-Alder reactions that need to be site-selective, etc.
    \item The website to help you predict Diels-Alders is historical at this point, so don't worry if you can't access it in the paper.
    \item There are several classes on computational chemistry in both Course 5 and Course 10 if you're interested!
    \item A problem with Reaxys: All of the reactions in the database are data-scraped from old papers, so a significant number of them are wrong or incomplete (20-30\%, and worse in other databases).
    \item Predictive modeling really reveals how difficult it is to predict reaction outcomes: Prof. Elkin has published papers where their model can predict yield far better than even chemistry experts.
    \item Conclusion: ML can be useful in predicting outcomes and can generalize to unseen reactions when descriptors carrying physically relevant information are used, and the machine gets appropriately formatted information.
    \item Note: None of this is testable material!
\end{itemize}



\section{Amines - 1}
\begin{itemize}
    \item \marginnote{10/25:}New lecturer for the second half of the course: Prof. Steve Buchwald.
    \begin{itemize}
        \item Born in Bloomington, Indiana.
        \item Undergrad at Brown, PhD at Harvard, Postdoc at Caltech (with Bob Grubbs, a Nobel laureate).
        \item At MIT for 40 years (since 1984).
        \item Has two cats :)
        \item Researches \textbf{organometallic chemistry}, with a focus on the synthesis of fine chemicals like pharmaceuticals.
        \begin{itemize}
            \item Most organometallic chemistry is predicated on the development of ligands.
            \item Many of Prof. Buchwald's ligands are named after his former cats!
            \item Example: The RuPhos ligand is named after Prof. Buchwald's since-passed cat, Rufus.
        \end{itemize}
    \end{itemize}
    \item \textbf{Organometallic} (chemistry): A hybrid of organic and inorganic chemistry.
    \item Prof. Elkin is in Washington, D.C. today advising the federal government!
    \pagebreak
    \item Announcements.
    \begin{itemize}
        \item The first half of this semester covered analytical techniques and physical chemistry; this half is more synthesis-focused.
        \item Review your 5.12 reactions!! A list of what you need to know for PSet 5 will be posted on Canvas.
        \begin{itemize}
            \item The teaching team will also keep a running list of reactions from this half of the course.
            \item This will tell you what to know for the exams and PSets.
        \end{itemize}
        \item Prof. Buchwald will post "study guides" for each unit, containing all the unit's content.
        \begin{itemize}
            \item \textcite{bib:Clayden} doesn't have a specific section on amines. Thus, the study guide lists all the pages spread throughout \textcite{bib:Clayden} where the different reactions can be found.
            \item If you still have \textcite{bib:Smith} --- your 5.12 textbook --- it's Chapter 23.
        \end{itemize}
        \item Plan: This lecture and the following one will cover amines.
        \begin{itemize}
            \item Amines have a special place in Prof. Buchwald's heart because they're connected to a lot of his research!
        \end{itemize}
        \item Like Prof. Elkin, Prof. Buchwald will continue giving fun facts that relate these topics to the real world.
    \end{itemize}
    \item Outline for the next two lectures.
    \begin{enumerate}[label={\Alph*.}]
        \item Intro.
        \item Chirality (or "handedness;" recall from 5.12).
        \item Br\o nsted basicity.
        \item Synthesis and reactivity (we'll spend the majority of our time on this topic).
        \begin{enumerate}[label={\arabic*.}]
            \item Alkylation of ammonia and alternatives.
            \item Reductive amination.
            \item Acylation and reduction.
            \item Reduction of nitriles (i.e., \ce{R-C#N} functional groups).
            \item Other miscellaneous methods.
        \end{enumerate}
    \end{enumerate}
    \item Today: We'll cover Topic A through most of Topic C.
    \item We now begin with Topic A: Introduction.
    \item \textbf{Amine}: An \ce{R3N} compound, where each \ce{R} may be distinct and \ce{R} is an \ce{H}, alkyl, or aryl group.
    \item The simplest amine is ammonia (\ce{NH3}).
    \begin{itemize}
        \item Notice that ammonia \emph{is} an amine by the definition: All of its \ce{R} groups are identically equal to \ce{H}!
        \item Fun fact: Ammonia is a necessary ingredient in fertilizer.
        \begin{itemize}
            \item It is prepared industrially from \ce{N2} using the Haber-Bosch process.
            \item One could make a reasonable argument that the industrial production of ammonia is the most important technological advance in the history of the world.
            \begin{itemize}
                \item This is because it enabled us to produce far more fertilizer, so that we could produce more food, so that we can feed a population of seven billion people.
            \end{itemize}
            \item Before Haber-Bosch, fertilizer came from an island covered in bird feces.
            \item Two Nobel prizes were awarded in connection with the development of this process.
            \begin{itemize}
                \item Haber won the Nobel Prize for his work on this process in 1918 (for the process).
                \item Bosch won the Nobel Prize for his work on this process in 1931 (for high-pressure chemistry).
            \end{itemize}
        \end{itemize}
    \end{itemize}
    \item Examples of amines.
    \begin{figure}[H]
        \centering
        \footnotesize
        \begin{subfigure}[b]{0.49\linewidth}
            \centering
            \chemfig{*6(-N=-(>:*5([:18]-N(-Me)----))=-=)}
            \caption{Nicotine.}
            \label{fig:amineExa}
        \end{subfigure}
        \begin{subfigure}[b]{0.49\linewidth}
            \centering
            \chemfig{H_2N-[:30]-[:-30]-[:30]-[:-30]-[:30]-[:-30]NH_2}
            \caption{Cadaverine.}
            \label{fig:amineExb}
        \end{subfigure}\\[2em]
        \begin{subfigure}[b]{0.49\linewidth}
            \centering
            \chemfig{N(-[:120]*6(=-=-=-))(-[:-120]*6(=-=-(-Me)=-))-*6(-=-(-*6(-=-(-N(-[:60]*6(-=-=-=))(-[:-60]*6(=-(-Me)=-=-)))=-=))=-=)}
            \caption{TPD.}
            \label{fig:amineExc}
        \end{subfigure}
        \begin{subfigure}[b]{0.49\linewidth}
            \centering
            \chemfig{Me-[:-30]-[:30](=[2]O)-[:-30]N(-[:30]*6(-=-=-=))-[6]*6(---N(--[:-150]-*6(-=-=-=))---)}
            \caption{Fentanyl.}
            \label{fig:amineExd}
        \end{subfigure}
        \caption{Amine examples.}
        \label{fig:amineEx}
    \end{figure}
    \begin{itemize}
        \item The top-selling pharmaceuticals in the world are all amines, at least in part.
        \begin{itemize}
            \item Not all of these "pharmaceuticals" are fun, though! Some are illicit drugs.
        \end{itemize}
        \item Example: Nicotine (Figure \ref{fig:amineExa}).
        \begin{itemize}
            \item It's one of the most difficult habits to break.
            \item There are drugs that mimic the structure of nicotine but bind to the receptor better and block nicotine from doing its job.
        \end{itemize}
        \item Example: Cadaverine (Figure \ref{fig:amineExb}).
        \begin{itemize}
            \item Does not smell good.
            \item When animals die, their flesh putrifies/rots and this is what causes the smell.
        \end{itemize}
        \item Example: TPD (Figure \ref{fig:amineExc}).
        \begin{itemize}
            \item This is a hole transport agent commonly found in the toner cartriges of laser printers.
        \end{itemize}
        \item Example: Fentanyl (Figure \ref{fig:amineExd}).
        \begin{itemize}
            \item A synthetic opioid that has caused unbelievable amounts of societal problems.
        \end{itemize}
    \end{itemize}
    \item Classes of amines.
    \begin{itemize}
        \item Ammonia (\ce{NH3}).
        \begin{itemize}
            \item Good because it helps feed the world.
            \item Bad because it's a toxic gas and smells horrible.
        \end{itemize}
        \item \textbf{Primary amines}.
        \item \textbf{Secondary amines}.
        \item \textbf{Tertiary amines}.
        \item \textbf{Quaternary ammonium salts}: A related family of comounds.
    \end{itemize}
    \pagebreak
    \item \textbf{Primary} (amine): An amine in which we've replaced one of the \ce{H}'s in ammonia with an (alkyl or aryl) \ce{R} group. \emph{Denoted by} $\bm{1^\circ}$. \emph{General form} \ce{RNH2}.
    \begin{figure}[h!]
        \centering
        \footnotesize
        \begin{subfigure}[b]{0.2\linewidth}
            \centering
            \ce{MeNH2}
            \caption{Methylamine.}
            \label{fig:amineEx1a}
        \end{subfigure}
        \begin{subfigure}[b]{0.2\linewidth}
            \centering
            \chemfig{NH_2-[4]*6(=-=-=-)}
            \caption{Aniline.}
            \label{fig:amineEx1b}
        \end{subfigure}
        \caption{Primary amine examples.}
        \label{fig:amineEx1}
    \end{figure}
    \begin{itemize}
        \item Example: Methylamine (Figure \ref{fig:amineEx1a}).
        \begin{itemize}
            \item A gas like ammonia, but a liquid under pressure.
            \item It's a controlled substance.
            \begin{itemize}
                \item In \emph{Breaking Bad}, this is what Walt, Jessie, and Todd heisted from the train!
            \end{itemize}
        \end{itemize}
        \item Example: Aniline (Figure \ref{fig:amineEx1b}).
        \begin{itemize}
            \item Very important historically: Modern chemistry began in the 1800's with aniline-based dyes.
            \begin{itemize}
                \item These companies are the precursor to modern-day pharmaceutical companies!
            \end{itemize}
        \end{itemize}
    \end{itemize}
    \item \textbf{Secondary} (amine): An amine in which we've replaced two of the \ce{H}'s in ammonia with (alkyl or aryl) \ce{R} groups. \emph{Denoted by} $\bm{2^\circ}$. \emph{General form} \ce{RR$'$NH}.
    \begin{figure}[h!]
        \centering
        \footnotesize
        \begin{subfigure}[b]{0.2\linewidth}
            \centering
            \chemfig{*6(-\chembelow{N}{H}-----)}\\[1em]
            \caption{Piperidine.}
            \label{fig:amineEx2a}
        \end{subfigure}
        \begin{subfigure}[b]{0.4\linewidth}
            \centering
            \schemestart
                \chemfig{NH(-[:120](-[::60]Me)(-[::-60]Me))(-[:-120](-[::60]Me)(-[::-60]Me))}
                \arrow
                \chemfig{NLi(-[:120](-[::60]Me)(-[::-60]Me))(-[:-120](-[::60]Me)(-[::-60]Me))}
            \schemestop
            \caption{Diisopropylamine and LDA.}
            \label{fig:amineEx2b}
        \end{subfigure}
        \caption{Secondary amine examples.}
        \label{fig:amineEx2}
    \end{figure}
    \begin{itemize}
        \item The \ce{R} groups can be separate, or they can be linked together.
        \item Example of a cyclic secondary amine: Piperidine (Figure \ref{fig:amineEx2a}).
        \begin{itemize}
            % \item Notice how the two alkyl groups are linked together to form a ring!
            \item Piperidine is important in a number of applications, including sequencing DNA.
        \end{itemize}
        \item Example of an acyclic secondary amine: Diisopropylamine (Figure \ref{fig:amineEx2b}).
        \begin{itemize}
            % \item Notice how the two isopropyl groups are not chemically bonded!
            \item If you replace the amine hydrogen with lithium, you get \underline{l}ithium \underline{d}iisopropyl\underline{a}mide (LDA).
            \begin{itemize}
                \item This is a very strong base that we'll talk more about later in this course.
            \end{itemize}
        \end{itemize}
    \end{itemize}
    \item \textbf{Tertiary} (amine): An amine in which we've replaced all three of the \ce{H}'s in ammonia with (alkyl or aryl) \ce{R} groups. \emph{Denoted by} $\bm{3^\circ}$. \emph{General form} \ce{RR$'$R$''$N}.
    \item \textbf{Quaternary ammonium salt}: A nitrogen covalently bonded to four \ce{R} groups (and hence having a positive formal charge), coordinated to a negative counterion. \emph{General form} \ce{R4N+ X-}.
    \begin{figure}[h!]
        \centering
        \footnotesize
        \chemfig{Me-[:30](=[2]O)-[:-30]O-[:30]-[:-30]-[:30]\charge{[extra sep=5pt]90=$\oplus$}{N}Me_3-[:160,0.7,,,opacity=0]\charge{45=$\ominus$}{X}}
        \caption{Quaternary ammonium salt example.}
        \label{fig:amineEx4}
    \end{figure}
    \begin{itemize}
        \item Example: Acetylcholine, an important neurotransmitter (Figure \ref{fig:amineEx4}).
        % \begin{itemize}
        %     \item This is an important neurotransmitter.
        % \end{itemize}
    \end{itemize}
    \pagebreak
    \item This concludes our introduction to amines.
    \item Aside: Prof. Buchwald \emph{strongly} recommends you show up for lecture the day before Halloween :)
    \item We now move onto Topic B: Chirality.
    \item Recall from 5.12 that some compounds are \emph{chiral}, i.e., they can have enantiomers.
    \begin{figure}[h!]
        \centering
        \footnotesize
        \chemfig{(-[2]R_4)(-[:-150]R_1)(<[:-60]R_2)(<:[:-20]R_3)}
        \caption{A chiral compound.}
        \label{fig:chiralC}
    \end{figure}
    \begin{itemize}
        \item These enantiomers can often be separated.
        \item They can also have different biological activities.
        \begin{itemize}
            \item Fun fact: The FDA now requires all chiral molecules to be prepared in both enantiomers and independently tested, in part because of the thalidomide scandal.
        \end{itemize}
    \end{itemize}
    \item The structure of amines.
    \begin{figure}[h!]
        \centering
        \vspace{2em}
        \footnotesize
        \chemfig{@{N}\charge{[extra sep=4.5mm]90=\:}{N}(-[:-150]R_1)(<[:-60]R_2)(<:[:-20]H)}
        \chemmove{
            \draw [orx,thick] ($(N)+(-0.0018,0.2)$) to[bend left=120,looseness=600] ($(N)+(0.0018,0.2)$) -- cycle;
        }
        \caption{Amine structure.}
        \label{fig:amineStruc}
    \end{figure}
    \begin{itemize}
        \item Amines are $sp^3$-hybridized with a tetrahedral electron pair arrangement.
        \begin{itemize}
            \item 3 bonding orbitals and 1 lone pair (lp).
        \end{itemize}
        \item The lp is responsible for the Br\o nsted basicity of amines.
        \item If one of the \ce{R} groups is hydrogen, then the amine can participate in hydrogen bonding (a very important interaction you should recall from Gen Chem).
    \end{itemize}
    \item Is pyridine a tertiary amine?
    \begin{itemize}
        \item Technically, yes; we'll discuss pyridine next lecture.
    \end{itemize}
    \item Amines have two enantiomers as well.
    \begin{figure}[h!]
        \centering
        \vspace{2em}
        \footnotesize
        \schemestart
            \chemfig{@{1N}\charge{[extra sep=4.5mm]90=\:}{N}(-[:-150]R_1)(<[:-60]R_2)(<:[:-20]R_3)}
            \arrow{<=>}
            \chemfig{@{2N}\charge{[extra sep=4.5mm]-90=\:}{N}(-[:150]R_1)(<[:20]R_2)(<:[:60]R_3)}
        \schemestop
        \chemmove{
            \draw [orx,thick] ($(1N)+(-0.0018,0.2)$) to[bend left=120,looseness=600] ($(1N)+(0.0018,0.2)$) -- cycle;
            \draw [orx,thick] ($(2N)+(-0.0018,-0.2)$) to[bend right=120,looseness=600] ($(2N)+(0.0018,-0.2)$) -- cycle;
        }\\[1.5em]
        \caption{Amine enantiomer interconversion.}
        \label{fig:amineEnaIntercon}
    \end{figure}
    \begin{itemize}
        \item The energy barrier ($\Delta G^\ddagger$) between the two enantiomers is \kcalr{5}{6}.
        \item Additionally, note that if $\Delta G^\ddagger\leq\kcal{20}$, the process is fast at room temperature.
        \item Thus, amine enantiomers rapidly interconvert at room temperature, so we (usually) cannot resolve amines into individual enantiomers.
        \begin{itemize}
            \item One time we can resolve amines into enantiomers is in the case of \textbf{aziridines}.
        \end{itemize}
    \end{itemize}
    \pagebreak
    \item \textbf{Aziridine}: A three-membered ring containing one nitrogen and two carbons. \emph{Structure}
    \begin{figure}[h!]
        \centering
        \footnotesize
        \chemfig[fixed length=false]{*3([:-30]--N(-R)-)}
        \caption{Aziridine.}
        \label{fig:aziridine}
    \end{figure}
    \begin{itemize}
        \item These are the amine equivalent of an epoxide.
        \item Like in any other amine, \ce{R} can still be \ce{H}, alkyl, or aryl.
        \item The $sp^3$-hybridized atoms all want to have \ang{109} bond angles but are strained to \ang{60}.
    \end{itemize}
    \item In order for aziridines to undergo \textbf{racemization}, the molecules must go through a transition state with an $sp^2$-nitrogen.
    \begin{figure}[h!]
        \centering
        \footnotesize
        \schemestart
            \chemfig{@{1N}\charge{[extra sep=4mm]54=\:}{N}?(-[:-54,0.7]Cl)-[:160](-[2,0.8]Ph)(-[:-150,0.8]Ph)-[:-120,0.5]?}
            \arrow
            \chemleft{[}
                \chemfig{@{2N}\charge{[extra sep=4.5mm]90=\:}{N}?(-[,0.7]Cl-[6,1.1,,,opacity=0])-[:160](-[2,0.8]Ph)(-[:-150,0.8]Ph)-[:-120,0.5]?}
            \chemright{]^\ddagger}
            % \arrow
            % \chemfig{@{3N}\charge{[extra sep=4mm]-54=\:}{N}?(-[:54,0.7]Cl)-[:160](-[2,0.8]Ph)(-[:-150,0.8]Ph)-[:-120,0.5]?}
        \schemestop
        \chemmove{
            \draw [orx,thick,rotate=-36] ($(1N)+(-0.0018,0.2)$) to[bend left=120,looseness=600] ($(1N)+(0.0018,0.2)$) -- cycle;
            \draw [orx,thick] ($(2N)+(-0.0018,0.2)$) to[bend left=120,looseness=600] ($(2N)+(0.0018,0.2)$) -- cycle;
            \filldraw [draw=orx,fill=ory,thick] ($(2N)+(-0.0018,-0.2)$) to[bend right=120,looseness=600] ($(2N)+(0.0018,-0.2)$) -- cycle;
            % \draw [orx,thick,rotate=36] ($(3N)+(-0.0018,-0.2)$) to[bend right=120,looseness=600] ($(3N)+(0.0018,-0.2)$) -- cycle;
        }
        \caption{Aziridine enantiomer interconversion.}
        \label{fig:aziridineEnaIntercon}
    \end{figure}
    \begin{itemize}
        \item This $sp^2$-nitrogen wants to have \ang{120} bond angles but is still strained down to \ang{60}.
        \begin{itemize}
            \item This is even worse than the strain in an $sp^3$-nitrogen!
        \end{itemize}
        \item Thus, the energy barrier to aziridine enantionmer interconversion is $\Delta G^\ddagger\approx\kcal{24}$.
        \item Therefore, (many) aziridines \emph{do not} interconvert at room temperature because $24>20$.
    \end{itemize}
    \item \textbf{Racemization}: The interconversion of enantiomers.
    \item This concludes our discussion of chirality.
    \item We now move onto Topic C: Br\o nsted basicity.
    \item Consider the following two protonation reactions.
    \begin{figure}[h!]
        \centering
        \footnotesize
        \begin{subfigure}[b]{0.3\linewidth}
            \centering
            \ce{MeOH + H+ <=> MeOH2+}
            \caption{Methanol.}
            \label{fig:baseMeOHNH2a}
        \end{subfigure}
        \begin{subfigure}[b]{0.3\linewidth}
            \centering
            \ce{MeNH2 + H+ <=> MeNH3+}
            \caption{Methylamine.}
            \label{fig:baseMeOHNH2b}
        \end{subfigure}
        \caption{Basicity of methanol vs. methylamine.}
        \label{fig:baseMeOHNH2}
    \end{figure}
    \begin{itemize}
        \item For \ce{MeOH2+}, $\pKa\approx -2$.
        \begin{itemize}
            \item This means that \ce{MeOH2+} is very acidic.
            \item It follows that \ce{MeOH} is only weakly basic.
        \end{itemize}
        \item For \ce{MeNH3+}, $\pKa\approx\numrange{9}{11}$.
        \begin{itemize}
            \item Thus, \ce{MeNH2} is \emph{much} more basic than \ce{MeOH}.
        \end{itemize}
    \end{itemize}
    \item Something critical to everyday life: Why do fish smell so bad after they die?
    \begin{figure}[h!]
        \centering
        \footnotesize
        \setcharge{extra sep=5pt}
        \schemestart
            \chemfig{Me_3\charge{90=$\oplus$}{N}-[,,,,->]\charge{[extra sep=3pt]45=$\ominus$}{O}}
            \arrow{->[Enzymes]}[,1.4]
            \chemfig{Me_3N}
            \arrow{->[\ce{H+}]}[,1.4]
            \chemfig{Me_3\charge{90=$\oplus$}{N}H}
        \schemestop
        \caption{Amines explain why fish smell, and how to season them!}
        \label{fig:amineFish}
    \end{figure}
    \begin{itemize}
        \item Not all fish smell to the same degree.
        \begin{itemize}
            \item Ocean fish (like cod) smell worse than river fish (like catfish) after they die.
        \end{itemize}
        \item Ocean fish smell worse because of trimethylamine oxide.
        \begin{itemize}
            \item There's a lot of salt in the ocean, so ocean fish use trimethylamine oxide to balance the salt levels in their cells.
            \item This compound does not smell very much, but after they die, enzymes from the fish (and from bacteria in the fish) reduce trimethylamine oxide to trimethylamine (which smells horrible).
        \end{itemize}
        \item Second important thing: We put lemon juice on fish because the acidity of the lemon juice (coming from citric acid) protonates the trimethylamine, decreasing the smell (and the taste since smell is connected to taste) so that the fish tastes better.
    \end{itemize}
    \item Resonance decreases the basicity of amines.
    \begin{figure}[h!]
        \centering
        \footnotesize
        \setcharge{extra sep=5pt}
        \schemestart
            \chemfig{*6([:-30]=-@{1C}=[@{12}](-[@{11}]@{1N}\charge{[extra sep=3pt]90=\:}{N}H_2)-=-)}
            \arrow{<->}
            \chemfig{*6([:-30]@{2C2}=[@{22}]-[@{21}]@{2C1}\charge{-60=$\ominus$}{}-(=\charge{90=$\oplus$}{N}H_2)-=-)}
            \arrow{<->}
            \chemfig{-[,0.4,,,opacity=0]*6(@{3C1}\charge{180=$\ominus$}{}-=-(=\charge{90=$\oplus$}{N}H_2)-@{3C2}=[@{32}]-[@{31}])}
            \arrow{<->}
            \chemfig{*6([:-30]-=-(=\charge{90=$\oplus$}{N}H_2)-\charge{60=$\ominus$}{}-=)}
        \schemestop
        \chemmove{
            \draw [curved arrow={5pt}{2pt}] (1N) to[bend right=90,looseness=3] (11);
            \draw [curved arrow={4pt}{3pt}] (12) to[bend right=90,looseness=4] (1C);
            \draw [curved arrow={10pt}{2pt}] (2C1) to[out=-60,in=-90,looseness=5] (21);
            \draw [curved arrow={4pt}{3pt}] (22) to[bend right=90,looseness=4] (2C2);
            \draw [curved arrow={10pt}{2pt}] (3C1) to[out=180,in=150,looseness=5] (31);
            \draw [curved arrow={4pt}{3pt}] (32) to[bend right=90,looseness=4] (3C2);
        }\\[1.5em]
        \caption{Basicity of aniline.}
        \label{fig:baseAniline}
    \end{figure}
    \begin{itemize}
        \item The conjugate base of aniline (\ce{PhNH3+}) has $\pKa\approx 5$, indicating that aniline is much less basic than methylamine ($\pKa\approx\numrange{9}{11}$).
        \item Why? Two reasons:
        \begin{enumerate}
            \item The $sp^2$-carbon adjacent to the nitrogen in aniline is more electoron-donating than the $sp^3$-carbon adjacent to the nitrogen in methylamine.
            \item Resonance.
            \begin{itemize}
                \item Just like in a phenol, we can push the heteroatom electrons into the benzene ring to get three other resonance forms (Figure \ref{fig:baseAniline}).
                \item Resonance decreases basicity, so aniline is much less basic than the any alkylamine.
            \end{itemize}
        \end{enumerate}
    \end{itemize}
\end{itemize}



\section{Amines - 2}
\begin{itemize}
    \item \marginnote{10/28:}Lecture 21 recap.
    \begin{enumerate}[label={\Alph*.}]
        \item Amines are basic, nitrogen-containing compounds.
        \begin{itemize}
            \item Their general form is \ce{R3N}, where $\ce{R}=\ce{H},\text{alkyl},\text{aryl}$.
            \item Some other types will be discussed at the end of the semester.
        \end{itemize}
        \item Types of amines: Ammonia (\ce{NH3}), $1^\circ$, $2^\circ$, or $3^\circ$ depending on the number of hydrogens.
        \item Amines are often chiral, but rarely resolvable.
        \item Amines are Br\o nsted bases.
        \begin{itemize}
            \item Substituents affect the acidities of the conjugate acids.
            \item You can compare the basicity of methylamine and aniline by comparing the $\pKa$'s of the conjugate acids.
            \item Resonance makes amines less basic.
        \end{itemize}
    \end{enumerate}
    \item Today: We'll cover Topic D.
    \begin{itemize}
        \item The reading --- \textcite[700-702]{bib:Clayden} --- covers snippets of amine synthesis.
    \end{itemize}
    \item We'll begin with Subtopic D.1: Alkylation of amines.
    \item Specifically, let's look at how we might synthesize a primary amine.
    \begin{figure}[H]
        \centering
        \footnotesize
        \schemestart
            \chemfig{H_3@{1N}\charge{90=\:}{N}}
            \arrow{0}[,0.5]
            \chemfig{-[:-150]@{2C}-[@{21}6]@{2Br}Br}
            \arrow
            \chemfig{H_3@{}\charge{[extra sep=5pt]90=$\oplus$}{N}(-[1,,,,opacity=0])(-[7,,,,opacity=0])-[:30]-[:-30]}
            \arrow(c3--c6){0}
            \chemfig{H_2@{}\charge{[extra sep=5pt]90=$\oplus$}{N}(-[1]-[::-60])(-[7]-[::60])}
            \arrow(--c9){0}
            \chemfig{@{}\charge{[extra sep=5pt]90=$\oplus$}{N}(-[1]-[::-60])(-[7]-[::60])(-[3]-[::60])(-[:-150]H)}
            \arrow(--c12){0}
            \chemfig{\charge{[extra sep=5pt]90=$\oplus$}{N}(-[1]-[::-60])(-[7]-[::60])(-[3]-[::60])(-[5]-[::-60])-[,0.5,,,opacity=0]\charge{45=$\ominus$}{Br}}
            \arrow(@c3--c4){<=>[*{0}\ce{NH3}]}[-90]
            \chemfig{H_2@{4N}\charge{90=\:}{N}(-[1,,,,opacity=0])(-[7,,,,opacity=0])-[:30]-[:-30]}
            \arrow{0[\scriptsize\chemfig[atom sep=1.4em]{-[:-150]@{5C}-[@{51}6]@{5Br}Br}]}
            \arrow(@c6--c7){<=>[*{0}\ce{NH3}]}[-90]
            \chemfig{H@{7N}\charge{90=\:}{N}(-[1]-[::-60])(-[7]-[::60])}
            \arrow{0[\scriptsize\chemfig[atom sep=1.4em]{-[:-150]@{8C}-[@{81}6]@{8Br}Br}]}
            \arrow(@c9--c10){<=>[*{0}\ce{NH3}]}[-90]
            \chemfig{@{10N}\charge{90=\:}{N}(-[1]-[::-60])(-[7]-[::60])(-[3]-[::60])}
            \arrow{0[\scriptsize\chemfig[atom sep=1.4em]{-[:-150]@{11C}-[@{111}6]@{11Br}Br}]}
        \schemestop
        \chemmove{
            \draw [shorten <=1em,shorten >=1em] (c4) -- ([xshift=-1.1cm]c4 -| c6) -- (c6);
            \draw [shorten <=1em,shorten >=1em] (c7) -- ([xshift=-1.5cm]c7 -| c9) -- (c9);
            \draw [shorten <=1em,shorten >=1em] (c10) -- ([xshift=-1.5cm]c10 -| c12) -- (c12);
            % 
            \draw [curved arrow={5pt}{2pt}] (1N) to[out=90,in=120,looseness=1.5] (2C);
            \draw [curved arrow={2pt}{2pt}] (21) to[bend left=90,looseness=2] (2Br);
            \draw [curved arrow={5pt}{2pt}] (4N) to[out=90,in=150] (5C);
            \draw [curved arrow={2pt}{2pt}] (51) to[bend left=90,looseness=2] (5Br);
            \draw [curved arrow={5pt}{2pt}] (7N) to[out=90,in=150] (8C);
            \draw [curved arrow={2pt}{2pt}] (81) to[bend left=90,looseness=2] (8Br);
            \draw [curved arrow={5pt}{2pt}] (10N) to[out=90,in=150] (11C);
            \draw [curved arrow={2pt}{2pt}] (111) to[bend left=90,looseness=2] (11Br);
        }
        \caption{Alkylation of amines from ammonia and an alkyl halide.}
        \label{fig:amineAlkylationRX}
    \end{figure}
    \begin{itemize}
        \item If we wanted to synthesize ethylamine (\ce{EtNH2}), we might first think to react ammonia with bromoethane via an S\textsubscript{N}2 mechanism.
        \item Would this work? Sort of.
        \begin{itemize}
            \item When we carry out this reaction, we obtain a primary ammonium cation that is easily (and reversibly) deprotonated to ethylamine by other basic ammonia molecules floating around.
            \item This frees up the ethylamine product to react again! In fact, even though ethylamine is sterically more hindered, it is electronically more activated.
            \item It follows that the ethylamine we've created will react \emph{even faster} than ammonia, forming a secondary ammonium cation.
        \end{itemize}
        \item After a few more successive cycles of S\textsubscript{N}2's and deprotonations --- creating iteratively more substituted and hence more electronically activated amines --- we obtain a quaternary ammonium salt\footnote{Note that at the board, Prof. Buchwald uses parentheses and numerical subscripts to indicate groups that are repeated multiple times.} as our major product.
        \item Therefore, the major product is tetraethylammonium, a quaternary ammonium salt.
    \end{itemize}
    \item Aside: When we do synthesis, we do \emph{not} want to form a mixture of products.
    \begin{itemize}
        \item Mixtures decrease our efficiency and require separation.
        \item We have all sorts ways to separate things, but separation techniques are inelegant, time consuming, and expensive.
    \end{itemize}
    \item As such, if we do want to use ammonia and an alkyl halide, we must use a \emph{large excess} of ammonia. However, this is not a great fix because\dots
    \begin{itemize}
        \item Ammonia is toxic and smells horrible;
        \item Ammonia is also a gas, and hence harder to control in the lab than a liquid.
    \end{itemize}
    \item So we need an alternate method to synthesize primary amines. In fact, we'll discuss two!
    \item Alternative \#1: Gabriel synthesis.
    \begin{figure}[H]
        \centering
        \footnotesize
        \schemestart
            \chemfig{*6(=-(*5(-(=O)-NH-(=O)-))=-=-)}
            \arrow{->[\ce{KOH}]}
            \chemfig{*6(=-(*5(-(=O)-@{2N}\charge{45=$\ominus$}{N}(-[,0.5,,,opacity=0]\charge{45=$\oplus$}{K})-(=O)-))=-=-)}
            \arrow{->[\scriptsize\chemfig[atom sep=1.4em]{-[:-150]@{3C}-[@{31}6]@{3Br}Br}]}
            \chemfig{*6(=-(*5(-(=O)-N(-[:30]-[:-30])-(=O)-))=-=-)}
            \arrow{-U>[\scriptsize\ce{H2N-NH2}][\scriptsize\chemfig[atom sep=1.4em]{H_2N-[:30]-[:-30]}][][0.4][80]}[,1.5]
            \chemfig{*6(=-(*6(-(=O)-NH-NH-(=O)-))=-=-)}
        \schemestop
        \chemmove{
            \draw [curved arrow={10pt}{2pt}] (2N) to[out=45,in=150] (3C);
            \draw [curved arrow={2pt}{2pt}] (31) to[bend left=90,looseness=2] (3Br);
        }
        \caption{Gabriel synthesis.}
        \label{fig:gabrielSynthesis}
    \end{figure}
    \begin{itemize}
        \item This method can be used to synthesize primary amines.
        \item The molecule we begin with is called phthalimide.
        \begin{itemize}
            \item Phthalimide has $\pKa\approx 8$.
            \item For comparison, \ce{NH3} has $\pKa\approx\numrange{33}{35}$.
        \end{itemize}
        \item First step: Put phthalimide in the presence of \ce{KOH} to yield the potassium salt.
        \item Second step: The potassium salt can do an S\textsubscript{N}2 reaction to monoalkylate.
        \begin{itemize}
            \item Importantly, this monoalkylated intermediate cannot react further! This is because its nitrogen lone pair is tied up in conjugation with the carbonyls.
        \end{itemize}
        \item Third step: We need to release the product, which we can do by adding hydrazine.
        \begin{itemize}
            \item This releases our desired ethylamine product and forms a byproduct.
            \item Aside: Hydrazine is also used as rocket fuel! It's an extremely high energy molecule.
        \end{itemize}
    \end{itemize}
    \item Alternative \#2: Reduction of azides.
    \begin{figure}[h!]
        \centering
        \footnotesize
        \setcharge{extra sep=5pt}
        \vspace{2.7em}
        \schemestart
            \chemfig{@{1R}R-[@{11}]@{1X}X}
            \arrow{->[\chemfig{\charge{90=$\oplus$}{Na}@{2N}\charge{90=$\ominus$}{N}_3}]}
            \chemfig{RN_3}
            \arrow{->[{[H]}]}
            \chemfig{RNH_2}
        \schemestop
        \chemmove{
            \draw [curved arrow={10pt}{2pt}] (2N) to[out=90,in=180,looseness=2] (1R);
            \draw [curved arrow={2pt}{2pt}] (11) to[bend left=90,looseness=3] (1X);
        }
        \caption{Reduction of azides.}
        \label{fig:azideReduce}
    \end{figure}
    \begin{itemize}
        \item This method can be used to synthesize primary \emph{or} secondary amines.
        \item We begin with an alkyl halide (\ce{RX}), where \ce{R} is primary or secondary.
        \begin{itemize}
            \item Importantly, \ce{R} \emph{cannot} be tertiary because the first step proceeds through an S\textsubscript{N}2 mechanism, and S\textsubscript{N}2 cannot happen with tertiary alkyl halides.
        \end{itemize}
        \item First step: We react \ce{RX} with sodium azide (\ce{NaN3}).
        \begin{itemize}
            \item Sodium azide is a source of azide (\ce{N3-}), a fantastic nucleophile.
            \item This will give us an \ce{RN3} intermediate.
        \end{itemize}
        \item Second step: We reduce the azide to the amine. There are two different ways to do this.\footnote{"\ce{[H]}" is a general way of denoting a reduction. It is useful in Figure \ref{fig:azideReduce} because there are two possible reducing agents we can use, discussed next.}
        \begin{itemize}
            \item Use lithium aluminum hydride (\ce{LiAlH4} \emph{or} LAH) followed by a water workup.
            \begin{itemize}
                \item Note: Whenever we use LAH, we need a water workup.
            \end{itemize}
            \item Use hydrogen gas (\ce{H2}) and palladium on carbon (Pd/C).
            \begin{itemize}
                \item Downside of these reagents: \ce{H2} is explosive, and it's a gas (recall from our discussion of ammonia earlier today that gases are harder to control).
            \end{itemize}
        \end{itemize}
        \item Downside of this method: \ce{RN3} is explosive, so it is too dangerous to run this process industrially.
        \begin{itemize}
            \item However, it's fine in small, controlled research settings when you know what you're doing.
        \end{itemize}
        \item Relevant reading: \textcite[354]{bib:Clayden}.
    \end{itemize}
    \item We now move onto Subtopic D.2: Reductive amination.
    \begin{itemize}
        \item Reductive amination is super useful!
        \begin{itemize}
            \item It is always in the \emph{Journal of Medicinal Chemistry}'s decadal list of the top 5 most common reactions used in their papers.
            \item Aside: Amide-bond formation is always (by far) the number 1 reaction, and a subject of Prof. Buchwald's research! It's not a perfectly solved problem, but we've gotten much better.
        \end{itemize}
        \item Relevant reading: \textcite[234-235]{bib:Clayden}.
    \end{itemize}
    \pagebreak
    \item Using reductive amination to convert secondary amines into tertiary amines.
    \begin{figure}[h!]
        \centering
        \footnotesize
        \schemestart
            \chemname[1.5em]{
                \chemfig{R-[:30](=[2]O)-[:-30]R'}
            }{carbonyl SM}
            \arrow(.20--){0}[,0]\+{,,-0.7em}
            \chemfig{R-[:-30]\chembelow{N}{H}-[:30]R}
            \arrow(--.173){->[\ce{Na(CN)BH3}][\ce{H+}]}[,1.7]
            \chemname{
                \chemleft{[}
                    \chemfig{R-[:30]@{3C}(=[@{31}2]@{3N}\charge{[extra sep=5pt]90=$\oplus$}{N}(-[:150]R)(-[:30]R))-[:-30]R'}
                \chemright{]}
            }{iminium ion}
            \arrow(.7--){->[\chemfig{@{4H}\charge{45=$\ominus$}{H}}]}[,1.7]
            \chemfig{R-[:30](-[2]N(-[:150]R)(-[:30]R))-[:-30]R'}
        \schemestop
        \chemmove{
            \draw [curved arrow={10pt}{3pt}] (4H) to[out=45,in=30,in looseness=2.5] (3C);
            \draw [curved arrow={3pt}{2pt}] (31) to[bend left=90,looseness=3] (3N);
        }
        \chemnameinit{}
        \caption{Reductive amination: $2^\circ\to 3^\circ$.}
        \label{fig:redAmin23}
    \end{figure}
    \begin{itemize}
        \item We begin with an aldehyde or a ketone (i.e., $\ce{R$'$}=\ce{H},\text{alkyl},\text{aryl}$).
        \item Single step: Use sodium cyanoborohydride (\ce{Na(CN)BH3}) in acidic medium.
        \item \ce{Na(CN)BH3} is a much milder, nicer reducing agent than sodium borohydride (\ce{NaBH4}).
        \begin{itemize}
            \item It selectively reduces \textbf{iminium ions} instead of the carbonyl starting material.
            \begin{itemize}
                \item This is important because if the carbonyl gets reduced to an alkane, it can no longer react with the secondary amine!
            \end{itemize}
            \item It is also stable under moderately acidic conditions.
            \begin{itemize}
                \item This is important because we don't want the acid to just neutralize our reducing agent.
            \end{itemize}
        \end{itemize}
        \item After the iminium ion is formed, hydride from \ce{Na(CN)BH3} attacks it. This yields the product.
        \item To reiterate: This is an incredibly powerful transformation.
    \end{itemize}
    \item Using reductive amination to convert primary amines into secondary amines.
    \begin{figure}[h!]
        \centering
        \footnotesize
        \begin{subfigure}[b]{\linewidth}
            \centering
            \schemestart
                \chemfig{R-[:30](=[2]O)-[:-30]R'}
                \arrow{0}[,0.1]\+
                \chemfig{RNH_2}
                \arrow(--.173){->[\ce{Na(CN)BH3}][\ce{H+}]}[,1.7]
                \chemfig{R-[:30](-[2]N(-[:150]R)(-[:30]H))(-[:-65]H)-[:-30]R'}
                \arrow{0}[90,0.8]
                \chemfig{R-[:30](=[2]O)-[:-30]R'}
                \merge>[\ce{Na(CN)BH3}][\ce{H+}](c3)(c4)--(.-163)[0.5,1.5,1]
                \chemfig{R-[:30](-[2]N(-[:150]R)(-[:30](-[:-30]R)(-[:55]H)-[2]R'))(-[:-65]H)-[:-30]R'}
            \schemestop
            \caption{Concurrent iminium formation and reduction.}
            \label{fig:redAmin12a}
        \end{subfigure}\\[2em]
        \begin{subfigure}[b]{\linewidth}
            \centering
            \schemestart
                \chemfig{R-[:30](=[2]O)-[:-30]H}
                \arrow{0}[,0.1]\+
                \chemfig{RNH_2}
                \arrow{->[\ce{H+}]}[,1.2]
                \chemname{
                    \chemfig{R-[:30](=[2]N-[:30]R')-[:-30]H}
                }{imine\hspace{5pt}}
                \arrow{->[{1. [H]\hspace{2mm}\ }][2. \ce{H2O}]}[,1.2]
                \chemfig{R-[:30](-[2,,,2]HN-[:30]R')-[:-30]H}
            \schemestop
            \chemnameinit{}
            \caption{Separate imine formation and reduction.}
            \label{fig:redAmin12b}
        \end{subfigure}
        \caption{Reductive amination: $1^\circ\to 2^\circ$.}
        \label{fig:redAmin12}
    \end{figure}
    \begin{itemize}
        \item Let's first try using the same conditions as in Figure \ref{fig:redAmin23}.
        \begin{itemize}
            \item If we do this, we run into the same problem as in Figure \ref{fig:amineAlkylationRX}.
            \item In particular, the product of the first reductive amination in Figure \ref{fig:redAmin12a} is a secondary amine and hence can react again to yield the rightmost product in Figure \ref{fig:redAmin12a}.
            \item Thus, if we did this, we'd have a mixture of products, and \emph{we do not like mixtures}!
        \end{itemize}
        \item Solution: Back off and run the reaction in two steps (Figure \ref{fig:redAmin12b}).
        \begin{itemize}
            \item First step: React an aldehyde with an amine to form an \textbf{imine}.
            \item Second step: Reduce the imine with either \ce{NaBH4} or \ce{LiAlH4}, followed by a water workup.
        \end{itemize}
        \item Aside: \ce{NaBH4} vs. \ce{LiAlH4}.
        \begin{itemize}
            \item Since \ce{NaBH4} is milder, we almost always prefer to use it over \ce{LiAlH4} when we can.
        \end{itemize}
        \item Aside: Why can't we use \ce{Na(CN)BH3}?
        \begin{itemize}
            \item Worse at reducing imines.
            \item More expensive than \ce{NaBH4}.
            \item Toxic (cyanide exposure).
        \end{itemize}
    \end{itemize}
    \item Using reductive amination to make a branched primary amine.
    \begin{figure}[h!]
        \centering
        \footnotesize
        \begin{subfigure}[b]{\linewidth}
            \centering
            \schemestart
                \chemfig{R-[:30](=[2]O)-[:-30]R'}
                \arrow{0}[,0.1]\+
                \chemfig{NH_3}
                \arrow{->[\ce{Na(CN)BH3}][\ce{H+}]}[,1.7]
                \chemleft{[}
                    \chemfig{R-[:30](=[2]NH)-[:-30]R'}
                \chemright{]}
                \arrow[,1.7]
                \chemfig{R-[:30](-[2]NH_2)-[:-30]R'}
            \schemestop
            \caption{Concurrent iminium formation and reduction.}
            \label{fig:redAmin01a}
        \end{subfigure}\\[2em]
        \begin{subfigure}[b]{\linewidth}
            \centering
            \schemestart
                \chemfig{R-[:30](=[2]O)-[:-30]R'}
                \arrow{->[\ce{H3NOH^+ Cl^-}]}[,1.7]
                \chemname{
                    \chemfig{R-[:30](=[2]N-[:30]OH)-[:-30]R'}
                }{oxime\hspace{2mm}\ }
                \arrow{->[1. \ce{LiAlH4}][2. \ce{H2O}\hspace{3.6mm}\ ]}[,1.7]
                \chemfig{R-[:30](-[2]NH_2)-[:-30]R'}
            \schemestop
            \chemnameinit{}
            \caption{Oxime formation and reduction.}
            \label{fig:redAmin01b}
        \end{subfigure}
        \caption{Reductive amination: Forming $1^\circ$.}
        \label{fig:redAmin01}
    \end{figure}
    \begin{itemize}
        \item Let's first try using the same conditions as in Figures \ref{fig:redAmin23} \& \ref{fig:redAmin12a}.
        \begin{itemize}
            \item If we do this, the bracketed imine intermediate proposed in Figure \ref{fig:redAmin01a} would be unstable.
            \item As such, we would need to resort to using a large excess of ammonia if we really want to make this work, even though such volumes are not ideal.
        \end{itemize}
        \item But what if you work in a place that doesn't allow you to handle gases?
        \item Solution: The two-step reaction in Figure \ref{fig:redAmin01b}.
        \begin{itemize}
            \item First step: Take your ketone or aldehyde and treat it with hydroxylamine hydrochloride (\ce{H3NOH^+ Cl^-}) to form an \textbf{oxime}.
            \begin{itemize}
                \item Unlike the proposed imine intermediate in Figure \ref{fig:redAmin01a}, oximes are \emph{really, really, really} stable.
            \end{itemize}
            \item Second step: Take the oxime and treat it with \ce{LiAlH4} followed by a water workup.
            \begin{itemize}
                \item Because oximes are so stable, we \emph{need} a really strong reducing agent like \ce{LiAlH4} to get the job done.
                \item More ways to reduce oximes are listed on \textcite[702,762,902]{bib:Clayden}.
            \end{itemize}
        \end{itemize}
        \item Thus, we obtain a gas-free synthetic route to branched primary amines.
    \end{itemize}
    \item We now move onto Subtopic D.3: Acylation and reduction.
    \begin{itemize}
        \item Acylation/reduction does monoalkylation, that is, the addition of one alkyl group to an amine.
        \begin{itemize}
            \item This may be $\ce{NH3}\to 1^\circ$, $1^\circ\to 2^\circ$, or $2^\circ\to 3^\circ$!
        \end{itemize}
        \item Reading on the acylation of amines, including the mechanism: \textcite[202-203]{bib:Clayden}.
        \item Reading on the reduction of amides: \textcite[531]{bib:Clayden}.
    \end{itemize}
    \item Example: Using acylation/reduction to convert primary amines to secondary amines.
    \begin{figure}[H]
        \centering
        \footnotesize
        \schemestart
            \chemfig{RNH_2}
            \+
            \chemfig{Cl-[:30](=[2]O)-[:-30]R'}
            \arrow{->[Base]}[,1.4]
            \chemfig{RHN-[:30](=[2]O)-[:-30]R'}
            \arrow{->[1. \ce{LiAlH4}][2. \ce{H2O}\hspace{3.6mm}\ ]}[,1.4]
            \chemfig{RHN-[:30]-[:-30]R'}
        \schemestop
        \caption{Monoalkylation by acylation and reduction.}
        \label{fig:acylReduce}
    \end{figure}
    \begin{itemize}
        \item We begin with an acid chloride and a primary amine.
        \item First step: Mix the starting materials with a base (such as \ce{Et3N}).
        \begin{itemize}
            \item This will form an amide.
            \item As in the Gabriel synthesis (see Figure \ref{fig:gabrielSynthesis}), this secondary amide does not react further because its nitrogen lone pair is tied up in conjugation with the carbonyl.
        \end{itemize}
        \item Second step: Reduce the amide with \ce{LiAlH4}, followed by a water workup.
        \begin{itemize}
            \item This affords the secondary amine product.
        \end{itemize}
    \end{itemize}
    \item Aside: Why do we need so many methods of making amines?
    \begin{itemize}
        \item Textbook chemistry (what we're doing) always works.
        \item In the lab, molecules have many properties that might get in the way of one method working, so we need alternatives to try.
        \begin{itemize}
            \item Example: Methods 1-26 might not work, but perhaps method 27 does.
        \end{itemize}
        \item This is the really exciting thing about Prof. Elkin's research: Prof. Elkin is using data science to avoid doing the first 26 bad reactions and make it so that the first time we try to do the reaction, it has a better chance of working.
    \end{itemize}
    \item We now move onto Subtopic D.4: Reduction of nitriles.
    \item The general form of this reaction is as follows.
    \begin{figure}[h!]
        \centering
        \footnotesize
        \schemestart
            \chemfig{R-C~N}
            \arrow{->[{[H]}]}
            \chemfig{RCH_2NH_2}
        \schemestop
        \caption{Reduction of nitriles.}
        \label{fig:nitrileReduc}
    \end{figure}
    \begin{itemize}
        \item The reducing agent can be \ce{LiAlH4}, or hydrogen and a nickel catalyst (\ce{H2}/Ni cat).
    \end{itemize}
    \item This reaction is pretty straightforward, but where did we get the nitrile from?
    \begin{figure}[h!]
        \centering
        \footnotesize
        \vspace{1.5em}
        \schemestart
            \chemfig{@{1R}R-[@{11}]@{1X}X}
            \arrow{->[\chemfig{@{2C}\charge{135=$\ominus$}{C}N}]}
            \chemfig{RCN}
            \arrow{->[{[H]}]}
            \chemfig{RCH_2NH_2}
        \schemestop
        \chemmove{
            \draw [curved arrow={10pt}{2pt}] (2C) to[out=135,in=180,looseness=2] (1R);
            \draw [curved arrow={2pt}{2pt}] (11) to[bend left=90,looseness=3] (1X);
        }
        \caption{Reduction of nitriles: Alkyl halide starting material.}
        \label{fig:nitrileReducX}
    \end{figure}
    \begin{itemize}
        \item Nitriles are often synthesized from (primary or secondary) alkyl halides through an S\textsubscript{N}2 reaction in which \ce{CN-} is the nucleophile.
        \begin{itemize}
            \item Once we have the nitrile, we can reduce it as in Figure \ref{fig:nitrileReduc}.
        \end{itemize}
        \item Therefore, the overall reaction in Figure \ref{fig:nitrileReducX} takes an alkyl halide to an amine with one additional \textbf{methylene} (\ce{CH2}) interspersed.
        \begin{itemize}
            \item This is called a \textbf{homologation} reaction, though you don't have to know that.
        \end{itemize}
    \end{itemize}
    \pagebreak
    \item Using the reduction of nitriles to synthesize 1,2-aminoalcohols.
    \begin{figure}[h!]
        \centering
        \footnotesize
        \schemestart
            \chemfig{*5([:18]---(=O)--)}
            \arrow{->[\ce{HCN}]}
            \chemfig{*5([:18]---(-[:70]CN)(-[:110]HO)--)}
            \arrow(--.-161){->[1. \ce{LiAlH4}][2. \ce{H2O}\hspace{3.6mm}\ ]}[,1.4]
            \chemfig{*5([:18]---(-[:60]-[:120]H_2N)(-[:120]HO)--)}
        \schemestop
        \caption{Reduction of nitriles: 1,2-aminoalcohol formation.}
        \label{fig:nitrileReduc12CNOH}
    \end{figure}
    \begin{itemize}
        \item We begin with a ketone.
        \item First step: Add \ce{HCN} to reduce the ketone to a \textbf{cyanohydrin}, a quasi-stable intermediate..
        \item Second step: Reduce the nitrile to afford the 1,2-aminoalcohol product.
        \item Why do we care about 1,2-aminoalcohols?
        \begin{itemize}
            \item Aside: Always ask why we care! Is it fundamentally interesting? Is there a practical application?
            \item In this case, 1,2-aminoalcohols are critical to a number of pharmaceuticals, so that's why we care about being able to synthesize them.
        \end{itemize}
        \item Reading on cyanohydrin formation: \textcite[127-29]{bib:Clayden}.
    \end{itemize}
    \item We now move onto Subtopic D.5: Miscellaneous reactions.
    \item The Hofmann rearrangement.
    \begin{figure}[h!]
        \centering
        \footnotesize
        \begin{subfigure}[b]{\linewidth}
            \centering
            \schemestart
                \chemfig{R-[:30]@{C}(=[2]@{O}O)-[:-30]NH_2}
                \arrow{->[\ce{NaOH}, \ce{H2O}][\ce{Cl2} or \ce{Br2}]}[,1.6]
                \chemfig{R-NH_2}
            \schemestop
            \chemmove{
                \draw [blx,-,line width=5mm,line cap=round,opacity=0.3] (C) -- (O.center);
            }
            \caption{General form.}
            \label{fig:HofmannRearra}
        \end{subfigure}\\[2em]
        \begin{subfigure}[b]{\linewidth}
            \centering
            \setcharge{extra sep=5pt}
            \schemestart
                \chemfig{R-[:30](=[2]O)-[:-30]@{1N}N(-[6]H)-[@{11}:30]@{1H}H}
                \arrow{<=>[\chemfig{@{2O}\charge{90=$\ominus$}{O}H}]}[,1.3]
                \chemfig{R-[:30](=[2]O)-[:-30]@{3N}\charge{90=$\ominus$}{N}(-[6]H)}
                \arrow{->[\chemfig[atom sep=1.4em]{@{4Cl}Cl-[@{41}]@{4O}OH}]}[,1.3]
                \chemfig{R-[@{53}:30](=[2]O)-[@{52}:-30]@{5N}N(-[@{51}6]@{5H}H)-[@{54}:30]@{5Cl}Cl}
                \arrow{->[*{0}\chemfig{@{6O}\charge{90=$\ominus$}{O}H}]}[-90]
                \chemname{
                    \chemfig{R-[:-30]N=[:30]@{7C}C=[:30]O}
                }{isocyanate}
                \arrow{->[\chemfig{@{8O}\charge{90=$\ominus$}{O}H}]}[180]
                \chemname{
                    \chemfig{R-[:-30]N-[:30](=[2]O)-[:-30]OH}
                }{carbamic acid}
                \arrow{-U>[][\ce{CO2}]}[180]
                \chemfig{R-NH_2}
            \schemestop
            \chemmove{
                \draw [curved arrow={9pt}{2pt}] (2O) to[bend right=90,looseness=2] (1H);
                \draw [curved arrow={2pt}{2pt}] (11) to[bend right=70,looseness=2] (1N);
                \draw [curved arrow={9pt}{1pt}] (3N) to[bend left=60,looseness=2] (4Cl);
                \draw [curved arrow={2pt}{2pt}] (41) to[bend left=80,looseness=2.5] (4O);
                \draw [curved arrow={9pt}{2pt}] (6O) to[out=90,in=-30,looseness=2] (5H);
                \draw [curved arrow={2pt}{2pt}] (51) to[out=30,in=30,looseness=2.2] (52);
                \draw [curved arrow={2pt}{2pt}] (53) to[bend right=40,looseness=1.2] (5N);
                \draw [curved arrow={2pt}{2pt}] (54) to[bend left=90,looseness=3] (5Cl);
                \draw [curved arrow={10pt}{2pt}] (8O) to[out=90,in=150] (7C);
            }
            \chemnameinit{}
            \caption{Mechanism.}
            \label{fig:HofmannRearrb}
        \end{subfigure}
        \caption{Hofmann rearrangement.}
        \label{fig:HofmannRearr}
    \end{figure}
    \begin{itemize}
        \item Figure \ref{fig:HofmannRearra} shows a very different kind of reaction from what we've seen.
        \begin{itemize}
            \item This reaction starts with a primary amide and involves reduction to a primary amine, excising the \ce{CO} highlighted in blue.
            \item \emph{Hint}: This reaction is related to the polymer problem on PSet 5!!
        \end{itemize}
        \item Reading: \textcite[1022]{bib:Clayden}.
        \item Let's now discuss the partial mechanism (Figure \ref{fig:HofmannRearrb}).
        \item First step: The base attacks an amide proton.
        \item Second step: The amide anion grabs a halogen from a hypohalous acid.
        \begin{itemize}
            \item Note that either hypochlorous acid (\ce{HOCl}) or hypobromous acid (\ce{HOBr}) will be formed \emph{in situ} from the reaction of the hydroxide base with \ce{Cl2} or \ce{Br2}, respectively.
            \item The acid function as an \ce{X+} equivalent, attracting the amide anion and leading to the formation of an \emph{N}-chloroamide intermediate.
            \item The amide halogen functions as an EWG, making the amide's remaining proton even more acidic than in the starting compound!
        \end{itemize}
        \item Third step: The extra-acidified \emph{N}-chloroamide proton get attacked by an equivalent of base, leading to a significant rearrangement step.
        \begin{itemize}
            \item This rearrangement produces an \textbf{isocyanate} intermediate.
        \end{itemize}
        \item Fourth step: The $sp$-hybridized carbon in the isocyanate reacts very rapidly to form a \textbf{carbamic acid} intermediate.
        \begin{itemize}
            \item As with "homologation" reactions, we won't ask you to name "carbamic acids" on an exam!!
        \end{itemize}
        \item Fifth step: The carbamic acid spontaneously loses \ce{CO2} to afford the amine.
    \end{itemize}
    \item Forming an aryl diazonium salt (\ce{ArN2^+ Cl^-}).
    \begin{figure}[h!]
        \centering
        \footnotesize
        \begin{subfigure}[b]{\linewidth}
            \centering
            \schemestart
                \chemfig{ArNH_2}
                \arrow{->[\ce{NaNO2}][\ce{HCl}]}[,1.2]
                \chemfig{Ar\charge{45=$\oplus$}{N}_2-[,0.4,,,opacity=0]\charge{45=$\ominus$}{Cl}}
            \schemestop
            \caption{General form.}
            \label{fig:diazoniumForma}
        \end{subfigure}\\[2em]
        \begin{subfigure}[b]{\linewidth}
            \centering
            \schemestart
                \chemfig{NaNO_2}
                \+
                \chemfig{HCl}
                \arrow
                \chemname{
                    \chemfig{HO-[:-30]N=[:30]O}
                }{HONO}
                \arrow(--.163){-U>[][\ce{H2O}][][0.3][80]}
                \chemname[-0.8em]{
                    \chemfig{N~\charge{45=$\oplus$}{O}}
                }{nitrosonium ion}
                \arrow(.17--){-U>[\ce{ArNH2}][][][0.3][80]}
                \chemfig{Ar\charge{45=$\oplus$}{N}_2-[,0.4,,,opacity=0]\charge{45=$\ominus$}{Cl}}
            \schemestop
            \caption{Mechanism.}
            \label{fig:diazoniumFormb}
        \end{subfigure}
        \caption{Aryl diazonium salt formation.}
        \label{fig:diazoniumForm}
    \end{figure}
    \begin{itemize}
        \item Reading: \textcite[520-23]{bib:Clayden}.
        \item First step: Sodium nitrite (\ce{NaNO2}) and \ce{HCl} form \textbf{HONO} \emph{in situ}.
        \item Second step: HONO loses water and forms the \textbf{nitrosonium ion} \emph{in situ}.
        \item Third step: The nitrosonium ion then reacts with aniline to do the nitration.
    \end{itemize}
    \item Next time (preview): A key reaction with aryl diazonium salts, related to the formation of an aryl diazonium salt from benzene.
    \begin{figure}[h!]
        \centering
        \footnotesize
        \schemestart
            \chemfig{*6(-=-=-=)}
            \arrow{->[\ce{HNO3}]}[,1.1]
            \chemfig{*6(-=-(-NO_2)=-=)}
            \arrow{->[\ce{H2}][\ce{Pd/C}]}[,1.1]
            \chemfig{*6(-=-(-NH_2)=-=)}
            \arrow{->[HONO]}[,1.1]
            \chemfig{*6(-=-(-\charge{45=$\oplus$}{N}_2-[0,0.4,,,opacity=0]\charge{45=$\ominus$}{Cl})=-=)}
        \schemestop
        \caption{Synthesizing an aryl diazonium salt from benzene.}
        \label{fig:diazoniumBenz}
    \end{figure}
\end{itemize}




\end{document}