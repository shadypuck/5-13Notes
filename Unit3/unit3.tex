\documentclass[../notes.tex]{subfiles}

\pagestyle{main}
\renewcommand{\chaptermark}[1]{\markboth{\chaptername\ \thechapter\ (#1)}{}}
\setcounter{chapter}{2}

\begin{document}




\chapter{Amines}
\setcounter{section}{19}
\section{Special Topics}
\begin{itemize}
    \item \marginnote{10/23:}Grade cutoffs on Exam 2.
    \begin{itemize}
        \item A-B cutoff: 80.
        \item B-C cutoff: 60.
        \item C-D cutoff: 45
        \item Only F's were people who did not take the exam.
        \item This was a significantly harder exam; y'all have been crushing it so far.
        \item Remember that these grades are meant to give you a perspective for what you're on track for; they are \emph{not} binding!
    \end{itemize}
    \item Notes on Steve Buchwald.
    \begin{itemize}
        \item He's a real big-name chemist: Has his name on a ton of reactions, can make a ton of pharmaceutical drugs, does a lot of consulting for chemical companies, etc.
        \item But also super kind, humble, and nice.
        \item Knows a ton, but is very down-to-earth and approachable.
    \end{itemize}
    \item The rest of this course will be much more synthesis-heavy.
    \begin{itemize}
        \item Feel free to continue to reach out to Prof. Elkin even though she's no longer at the blackboards!
    \end{itemize}
    \item Today: We'll have fun and talk about machine learning.
    \begin{itemize}
        \item Prof. Elkin will go through \textcite{bib:ML}, a paper about using machine learning to predict the outcome of Diels-Alder reactions.
    \end{itemize}
    \item The basic idea of what the authors are saying is that if you encode the substituents, you get good prediction of the outputs!
    \begin{itemize}
        \item Your computer doesn't know what a molecule is, so you have to encode your molecule in a way that is meaningful to a computer.
        \item For example: You should not encode benzene with alternating single- and double bonds; benzene has six equivalent bonds due to resonance!
    \end{itemize}
    \item Nowadays, computers can predict biological activities (doesn't work perfectly yet, though great progress), solubility and crystal structures (works fine), NMR spectra (works awesome), etc.
    \begin{itemize}
        \item Predicting optimal reaction conditions works awesome.
    \end{itemize}
    \item Predicting reaction outcomes or yields can be hit or miss.
    \item There have been maybe \num{1000000} chemical reactions ever catalogued, but most of them are not that useful.
    \item The low-data regime of predictive modeling is the final frontier, and the especially important one for chemistry.
    \item Taking high-level expertise and making it algorithmically applicable can be really difficult.
    \item "High accuracies are achieved only if the machine is provided some chemical `insight' about the reaction (in particular, information about the reaction's core and key substituents)."
    \item While ML models cannot provide the generality of quantum mechanics, they work much faster.
    \item They trained the model with inverse electron-demand Diels-Alder reactions, Diels-Alder reactions that need to be site-selective, etc.
    \item The website to help you predict Diels-Alders is historical at this point, so don't worry if you can't access it in the paper.
    \item There are several classes on computational chemistry in both Course 5 and Course 10 if you're interested!
    \item A problem with Reaxys: All of the reactions in the database are data-scraped from old papers, so a significant number of them are wrong or incomplete (20-30\%, and worse in other databases).
    \item Predictive modeling really reveals how difficult it is to predict reaction outcomes: Prof. Elkin has published papers where their model can predict yield far better than even chemistry experts.
    \item Conclusion: ML can be useful in predicting outcomes and can generalize to unseen reactions when descriptors carrying physically relevant information are used, and the machine gets appropriately formatted information.
    \item Note: None of this is testable material!
\end{itemize}




\end{document}