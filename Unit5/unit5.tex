\documentclass[../notes.tex]{subfiles}

\pagestyle{main}
\renewcommand{\chaptermark}[1]{\markboth{\chaptername\ \thechapter\ (#1)}{}}
\setcounter{chapter}{4}

\begin{document}




\chapter{Enolate Chemistry}
\setcounter{section}{28}
\section{Enols and Enolates}
\begin{itemize}
    \item \marginnote{11/15:}Grade cutoffs on Exam 3.
    \begin{itemize}
        \item A: 85-100.
        \item B: 70-84.
        \item C: 63-69.
        \item $<\text{C}$: $<57$.
        \item If you are considering dropping this class, the drop date is 11/20.
        \begin{itemize}
            \item It does not count as a drop if you just stop showing up and stop submitting assignments.
            \item Go to the Registrar's site and fill out an add/drop form.
        \end{itemize}
        \item If you are doing less well than you had hoped or expected, talk to your TFs about options!
        \begin{itemize}
            \item You may be eligible for tutoring.
            \item It is \emph{your responsibility} to reach out for help.
        \end{itemize}
    \end{itemize}
    \item Fun (or scary) Friday: Prof. Buchwald sings the elements song!
    \item Announcement: Unit 5 study guide posted.
    \item We now begin the first of four lectures in Unit 5: Enols and enolates.
    \begin{itemize}
        \item Readings: Chapters 20, 25, 26 of \textcite{bib:Clayden}.
    \end{itemize}
    \item Lecture outline.
    \begin{enumerate}[label={\Alph*.}]
        \item Background.
        \begin{itemize}
            \item Enolate definition.
            \item Keto-enol tautomerization (base-catalyzed and acid-catalyzed).
            \item Evidence: Deuterium exchange.
        \end{itemize}
        \item $\alpha$-halogenation of ketones.
        \begin{itemize}
            \item Base-promoted mechanism (and complications).
            \item The haloform reaction.
            \item Acid-catalyzed mechanism.
        \end{itemize}
        \item $\alpha$-alkylation.
        \begin{itemize}
            % \item General form.
            \item Lithium diisopropylamide.
            \item Malonate ester synthesis.
            \item Kinetic vs. thermodynamic enolates.
        \end{itemize}
    \end{enumerate}
    \pagebreak
    \item We'll begin with Topic A: Background.
    \item Defining enolates.
    \begin{figure}[h!]
        \centering
        \begin{subfigure}[b]{0.4\linewidth}
            \centering
            \includegraphics[width=0.23\linewidth]{carbonylRxna.png}
            \caption{As electrophile.}
            \label{fig:carbonylRxna}
        \end{subfigure}
        \begin{subfigure}[b]{0.4\linewidth}
            \centering
            \includegraphics[width=0.9\linewidth]{carbonylRxnb.png}
            \caption{As nucleophile.}
            \label{fig:carbonylRxnb}
        \end{subfigure}
        \caption{Carbonyl-based chemical reactions.}
        \label{fig:carbonylRxn}
    \end{figure}
    \begin{itemize}
        \item Carbonyls have two important modes of reactivity.
        \item We've already discussed how carbonyls can act as electrophiles (Figure \ref{fig:carbonylRxna}).
        \begin{itemize}
            \item This yields a tetrahedral intermediate, as we've discussed.
        \end{itemize}
        \item The other mode of reactivity --- which is new and our focus --- is that we can deprotonate at the $\alpha$-carbon to make a nucleophilic species (Figure \ref{fig:carbonylRxnb}).
        \begin{itemize}
            \item The major resonance structure will be the oxygen-centered one (because oxygen is more electronegative).
            \item However, most reactions we're interested in proceed at carbon.
        \end{itemize}
    \end{itemize}
    \item Key concept: Oxygen \emph{enables} this mode of reactivity stabilizing the negative charge.
    \begin{figure}[h!]
        \centering
        \begin{subfigure}[b]{0.25\linewidth}
            \centering
            \includegraphics[width=0.38\linewidth]{alkeneCarbonyla.png}
            \caption{Not as electrophiles.}
            \label{fig:alkeneCarbonyla}
        \end{subfigure}
        \begin{subfigure}[b]{0.25\linewidth}
            \centering
            \includegraphics[width=0.38\linewidth]{alkeneCarbonylb.png}
            \caption{Not as nucleophiles.}
            \label{fig:alkeneCarbonylb}
        \end{subfigure}
        \caption{Alkenes do not react via carbonyl-analogous pathways.}
        \label{fig:alkeneCarbonyl}
    \end{figure}
    \begin{itemize}
        \item For the purposes of 5.13, analogous addition to alkenes (Figure \ref{fig:alkeneCarbonyla}) and $\alpha$-deprotonation of alkenes (Figure \ref{fig:alkeneCarbonylb}) is very rare.
    \end{itemize}
    \item Let's now discuss tautomers.
    \begin{figure}[h!]
        \centering
        \includegraphics[width=0.32\linewidth]{ketoEnol.png}
        \caption{Keto-enol tautomerization.}
        \label{fig:ketoEnol}
    \end{figure}
    \begin{itemize}
        \item Ketones can tautomerize to \textbf{enols} (a portmanteau of alk\underline{en}e and alcoh\underline{ol}).
        \item The keto and enol form are known as \textbf{tautomers}.
        \item The equilibrium favors the keto form by far (about a million to one; we'll only have $0.001\%$ enol).
    \end{itemize}
    \item Catalysts can speed up the inverconversion, but they can't change the equilibrium.
    \begin{itemize}
        \item Let's discuss the mechanism by which bases and acids speed this process up, though.
    \end{itemize}
    \pagebreak
    \item Base-catalyzed keto-enol tautomerization mechanism.
    \begin{figure}[h!]
        \centering
        \includegraphics[width=0.55\linewidth]{ketoEnolBase.png}
        \caption{Keto-enol tautomerization mechanism (base-catalyzed).}
        \label{fig:ketoEnolBase}
    \end{figure}
    \begin{itemize}
        \item The $\alpha$-carbon of a ketone has $\pKa\approx 20$.
        \begin{itemize}
            \item This is a good number to memorize, not because you'll ever be tested on it but because understanding relative $\pKa$'s will aid your chemical intuition.
        \end{itemize}
        \item Hydroxide can speed up this process by deprotonating the $\alpha$-carbon.
        \begin{itemize}
            \item Then we just protonate the oxygen.
        \end{itemize}
        \item Recall that we still have $\Keq\ll 1$.
    \end{itemize}
    \item Acid-catalyzed keto-enol tautomerization mechanism.
    \begin{figure}[h!]
        \centering
        \begin{subfigure}[b]{\linewidth}
            \centering
            \includegraphics[width=0.62\linewidth]{ketoEnolAcida.png}
            \caption{Correct mechanism.}
            \label{fig:ketoEnolAcida}
        \end{subfigure}\\[2em]
        \begin{subfigure}[b]{\linewidth}
            \centering
            \includegraphics[width=0.4\linewidth]{ketoEnolAcidb.png}
            \caption{Incorrect mechanism.}
            \label{fig:ketoEnolAcidb}
        \end{subfigure}
        \caption{Keto-enol tautomerization mechanism (acid-catalyzed).}
        \label{fig:ketoEnolAcid}
    \end{figure}
    \begin{itemize}
        \item We can either write the reagents equivalently as \ce{H+}/\ce{H2O} or \ce{H3O+}.
        \item As we've been doing, we begin by protonating the carbonyl.
        \item Then the best base in solution comes and deprotonates the $\alpha$-carbon.
        \begin{itemize}
            \item Water isn't a great base, but it's all we've got.
        \end{itemize}
        \item Note that we do \emph{not} do deprotonation first and protonation second, as drawn in Figure ??b.
        \begin{itemize}
            \item Remember that anions cannot exist in acidic solution!
        \end{itemize}
    \end{itemize}
    \item So this is all great, but what if we don't believe Prof. Buchwald that tautomerization occurs?
    \begin{itemize}
        \item It's good to question things in science!
        \item Many times, we've assumed things that later experiments have proven incorrect.
    \end{itemize}
    \pagebreak
    \item We can find evidence for enolization via an isotopic labeling study.
    \begin{figure}[h!]
        \centering
        \includegraphics[width=0.4\linewidth]{ketoEnolIsotope.png}
        \caption{Isotopic labeling provides evidence for keto-enol tautomerization.}
        \label{fig:ketoEnolIsotope}
    \end{figure}
    \begin{itemize}
        \item If we dissolve acetone in basic deuterated water and deuteroxide (or acid), we will eventually obtain deuteroacetone.
        \item The mechanism proceeds analogously to Figure \ref{fig:ketoEnolBase} or \ref{fig:ketoEnolAcida}, except that our reagents are all \ce{DO-} and \ce{D2O}.
        \begin{itemize}
            \item In particular, we replace each of the six hydrogens one at a time with deuterium, eventually leading to the product.
            \item We form the fully deuterated product instead of a \ce{H}/\ce{D}-mixed product because we assume that the concentration of deuterated acid or base and water is \emph{much} greater than the concentration of acetone. This is similar to the swamping effect in Figure \ref{fig:transesterEqa}.
        \end{itemize}
    \end{itemize}
    \item We now move onto Topic B: $\alpha$-halogenation of ketones.
    \begin{itemize}
        \item We can do this with chlorine, bromine, or iodine.
    \end{itemize}
    \item Base-promoted $\alpha$-halogenation mechanism.
    \begin{figure}[h!]
        \centering
        \includegraphics[width=0.8\linewidth]{alphaHaloBase.png}
        \caption{$\alpha$-halogenation mechanism (base-promoted).}
        \label{fig:alphaHaloBase}
    \end{figure}
    \begin{itemize}
        \item Imagine we mix cyclohexanone with chlorine gas under basic conditions. What's going to happen?
        \item We'll form a small amount of enolate, and then chlorinate to form $\alpha$-chlorocyclohexanone.
        \begin{itemize}
            \item We declare victory!
            \item Except that the world is a harsh place and --- like in Figure \ref{fig:redAmin12a} --- we can get further reactivity.
        \end{itemize}
        \item In particular, the hydrogen geminal to the $\alpha$-chlorine is now \emph{more} acidic (proximity to an EWG, so anion is stabilized).
        \begin{itemize}
            \item Thus, we can react again to get $\alpha$-dichlorocyclohexanone.
        \end{itemize}
        \item Thus, this reaction is not good\dots except in one case.
    \end{itemize}
    \pagebreak
    \item The iodoform reaction.
    \begin{figure}[h!]
        \centering
        \includegraphics[width=0.7\linewidth]{iodoform.png}
        \caption{Iodoform reaction.}
        \label{fig:iodoform}
    \end{figure}
    \begin{itemize}
        \item In the first step, we have three successive iodinations to yield the triiodomethylketone.
        \item This is such a strong EWG and good leaving group that the triiodomethylketone acts kind of like an acid chloride.
        \begin{itemize}
            \item In particular, we get an addition-elimination mechanism that kicks out the triiodomethanide anion.
            \item This anion can then be protonated by the resultant carboxylic acid to yield iodoform (\ce{HCI3}) and a stable carboxylate.
        \end{itemize}
        \item Iodoform precipitates as a yellow solid.
        \begin{itemize}
            \item In the olden days, it used to be a test for a ketone.
            \item Before we had NMR, mass spec, and other kinds of spectroscopy, we had a bunch of test reagents that we would add to our compounds to determine what it was.
            \item Essentially, if we had a compound and we didn't know what it was but thought it was a ketone, we could confirm or deny this by adding iodine and base to our mixture!
        \end{itemize}
    \end{itemize}
    \item What does it mean when Prof. Buchwald draws a circular arrow from a carbonyl $\pi$-bond back to it?
    \begin{itemize}
        \item They use this in \textcite{bib:Clayden}!
        \item This is a shorthand for the two-step addition-elimination process, in which electrons kick up in a first step and then kick back down in a second step.
        \item This is similar to how we shorthand a two-step proton transfer as "PT!"
    \end{itemize}
    \item So how do we make mono-$\alpha$-haloketones, if that's our goal?
    \begin{itemize}
        \item Use acid-catalyzed $\alpha$-halogenation!
    \end{itemize}
    \item Acid-catalyzed $\alpha$-halogenation mechanism.
    \begin{figure}[h!]
        \centering
        \includegraphics[width=\linewidth]{alphaHaloAcid.png}
        \caption{$\alpha$-halogenation mechanism (acid-catalyzed).}
        \label{fig:alphaHaloAcid}
    \end{figure}
    \begin{itemize}
        \item Acids encourage the rate of formation of the enol.
        \item Then if we do this in the presence of bromine, we'll get $\alpha$-bromoacetone (following deprotonation).
        \item Now the product is \emph{less} reactive than the starting material (because the bromine EWG stabilizes the carbonyl and disfavors protonation of it).
        \item Takeaway: Acid-catalyzed $\alpha$-halogenation is selective for monohalogenation.
        \item This process is used to synthesize a lot of medicines and drug molecules.
    \end{itemize}
    \pagebreak
    \item We now move onto Topic C: $\alpha$-alkylation.
    \begin{itemize}
        \item This is the heavy hitter; a really, really important reaction of ketones.
    \end{itemize}
    \item General form.
    \begin{figure}[h!]
        \centering
        \includegraphics[width=0.65\linewidth]{alphaAlk.png}
        \caption{$\alpha$-alkylation.}
        \label{fig:alphaAlk}
    \end{figure}
    \begin{itemize}
        \item Suppose we want to convert a ketone into a new compound where we've formed a \ce{C-C} bond.
        \item The other reagent is a primary or secondary alkyl halide.
    \end{itemize}
    \item Drawing a mechanism for this doesn't seem too bad at first.
    \begin{itemize}
        \item We may deprotonate to the enolate and attack the alkyl halide to start.
        \item But there is a complication.
        \begin{itemize}
            \item We get lots of side reactions!
        \end{itemize}
        \item In 5.13, we're all about efficiency and elegance, so this is not good.
    \end{itemize}
    \item There are several solutions to this issue, which we'll discuss presently.
    \item Solution 1: Use lithium diisopropylamide (LDA).
    \begin{figure}[h!]
        \centering
        \includegraphics[width=0.6\linewidth]{alphaAlkLDA.png}
        \caption{$\alpha$-alkylation with lithium diisopropylamide.}
        \label{fig:alphaAlkLDA}
    \end{figure}
    \begin{itemize}
        \item See Figure \ref{fig:amineEx2b} for the structure and synthesis of LDA.
        \item Helpful characteristics of LDA.
        \begin{itemize}
            \item LDA is a strong base.
            \item It is secondary and hence hindered (therefore a poor nucleophile).
            \item The conjugate acid of LDA has $\pKa\approx 35$.
            \item Thus, it will only deprotonate and not do any competitive addition chemistry!
        \end{itemize}
        \item We begin with an essentially irreversible deprotonation to the enolate.
        \item This is followed by 100\% conversion to the alkylated product.
    \end{itemize}
    \item Using LDA is a relatively modern solution --- only about 50 years old.
    \begin{itemize}
        \item However, organic chemistry has been around for close to 250 years!
        \item The roots of organic chemistry are in the old German dye industry, which morphed into the present-day pharmaceutical industry.
        \item So how did people do this stuff before LDA? Via solution 2.
    \end{itemize}
    \pagebreak
    \item Solution 2: Malonate ester synthesis.
    \begin{figure}[h!]
        \centering
        \begin{subfigure}[b]{\linewidth}
            \centering
            \includegraphics[width=0.6\linewidth]{alphaAlkMalEstera.png}
            \caption{Malonate ester resonance.}
            \label{fig:alphaAlkMalEstera}
        \end{subfigure}\\[2em]
        \begin{subfigure}[b]{\linewidth}
            \centering
            \includegraphics[width=0.5\linewidth]{alphaAlkMalEsterb.png}
            \caption{Regular ester resonance.}
            \label{fig:alphaAlkMalEsterb}
        \end{subfigure}
        \caption{$\alpha$-alkylation with malonate esters.}
        \label{fig:alphaAlkMalEster}
    \end{figure}
    \begin{itemize}
        \item The starting material has esters on both sides (either ethyl or methyl; it doesn't matter).
        \item The important thing is that for the malonate ester, $\pKa\approx 13$.
        \begin{itemize}
            \item In contrast, a regular ester has $\pKa\approx 25$.
        \end{itemize}
        \item Why this drastic difference in $\pKa$?
        \begin{itemize}
            \item The deprotonated malonate ester's anion has more resonance forms (two adjacent carbonyls into which to delocalze!) than the deprotonated ester (only one adjacent carbonyl).
        \end{itemize}
        \item This difference leads us to call the deprotonated malonate ester a \textbf{soft enolate}.
        \begin{itemize}
            \item These characteristics make it very easy and safe to work with, so it's often used at scale.
        \end{itemize}
    \end{itemize}
    \item We'll now quickly introduce a topic that we'll also discuss more next time.
    \item Kinetic vs. thermodynamic enolates.
    \item \textbf{Kinetic} (enolate): The enolate generated by deprotonation at the less-substituted position, all else being equal.
    \item Example: LDA (really big and bulky) will selectively form the kinetic enolate at the unsubstituted position of $\alpha$-methylcyclohexanone.
    \begin{figure}[h!]
        \centering
        \includegraphics[width=0.6\linewidth]{enolateKinetic.png}
        \caption{Kinetic enolate formation.}
        \label{fig:enolateKinetic}
    \end{figure}
    \begin{itemize}
        \item This enolate could then be used --- for example --- to attack methyl iodide (\ce{MeI}) and alkylate.
        \item Note that this process would most likely form a mixture of stereoisomers.
    \end{itemize}
    \pagebreak
    \item \textbf{Thermodynamic} (enolate): The enolate that is more stable.
    \item Example: Potassium \emph{t}-butoxide (\ce{KO{}^{\emph{t}}Bu}) has $\pKa\approx\text{\numrange{16}{18}}$, so it deprotonates $\alpha$-methylcyclohexanone reversibly until we get the more stable one.
    \begin{figure}[h!]
        \centering
        \includegraphics[width=0.55\linewidth]{enolateThermodynamic.png}
        \caption{Thermodynamic enolate formation.}
        \label{fig:enolateThermodynamic}
    \end{figure}
    \begin{itemize}
        \item Treating this with \ce{MeI} then generates the $\alpha$-dimethylated form of cyclohexanone.
    \end{itemize}
    \item You can add in \ce{Me3SiCl} to trap enolates the silyl enol ether.
    \begin{figure}[h!]
        \centering
        \includegraphics[width=0.75\linewidth]{enolateTMS.png}
        \caption{Trapping enolates as silyl enol ethers.}
        \label{fig:enolateTMS}
    \end{figure}
    \begin{itemize}
        \item This silyl protecting group could then be removed with \ce{MeLi}, regenerating the enolate and yielding tetramethylsilane (\ce{SiMe4}) as a byproduct.
        \item In the deprotection step, the methyl anion attacks the silicon atom in the TMS group, engaging in an S\textsubscript{N}2 displacement.
    \end{itemize}
\end{itemize}




\end{document}